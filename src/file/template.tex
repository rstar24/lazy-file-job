\documentclass{report}
\usepackage{listings}
%%%%%%%%%%%%%%%%%%%%%%%%%%%%%%%%%
% PACKAGE IMPORTS
%%%%%%%%%%%%%%%%%%%%%%%%%%%%%%%%%


\usepackage[tmargin=2cm,rmargin=1in,lmargin=1in,margin=0.85in,bmargin=2cm,footskip=.2in]{geometry}
\usepackage{amsmath,amsfonts,amsthm,amssymb,mathtools}
\usepackage[varbb]{newpxmath}
\usepackage{xfrac}
\usepackage[makeroom]{cancel}
\usepackage{mathtools}
\usepackage{bookmark}
\usepackage{enumitem}
\usepackage{hyperref,theoremref}
\hypersetup{
	pdftitle={Assignment},
	colorlinks=true, linkcolor=doc!90,
	bookmarksnumbered=true,
	bookmarksopen=true
}
\usepackage[most,many,breakable]{tcolorbox}
\usepackage{xcolor}
\usepackage{varwidth}
\usepackage{varwidth}
\usepackage{etoolbox}
%\usepackage{authblk}
\usepackage{nameref}
\usepackage{multicol,array}
\usepackage{tikz-cd}
\usepackage[ruled,vlined,linesnumbered]{algorithm2e}
\usepackage{comment} % enables the use of multi-line comments (\ifx \fi) 
\usepackage{import}
\usepackage{xifthen}
\usepackage{pdfpages}
\usepackage{transparent}

\newcommand\mycommfont[1]{\footnotesize\ttfamily\textcolor{blue}{#1}}
\SetCommentSty{mycommfont}
\newcommand{\incfig}[1]{%
    \def\svgwidth{\columnwidth}
    \import{./figures/}{#1.pdf_tex}
}

\usepackage{tikzsymbols}
\renewcommand\qedsymbol{$\Laughey$}


%\usepackage{import}
%\usepackage{xifthen}
%\usepackage{pdfpages}
%\usepackage{transparent}


%%%%%%%%%%%%%%%%%%%%%%%%%%%%%%
% SELF MADE COLORS
%%%%%%%%%%%%%%%%%%%%%%%%%%%%%%



\definecolor{myg}{RGB}{56, 140, 70}
\definecolor{myb}{RGB}{45, 111, 177}
\definecolor{myr}{RGB}{199, 68, 64}
\definecolor{mytheorembg}{HTML}{F2F2F9}
\definecolor{mytheoremfr}{HTML}{00007B}
\definecolor{mylenmabg}{HTML}{FFFAF8}
\definecolor{mylenmafr}{HTML}{983b0f}
\definecolor{mypropbg}{HTML}{f2fbfc}
\definecolor{mypropfr}{HTML}{191971}
\definecolor{myexamplebg}{HTML}{F2FBF8}
\definecolor{myexamplefr}{HTML}{88D6D1}
\definecolor{myexampleti}{HTML}{2A7F7F}
\definecolor{mydefinitbg}{HTML}{E5E5FF}
\definecolor{mydefinitfr}{HTML}{3F3FA3}
\definecolor{notesgreen}{RGB}{0,162,0}
\definecolor{myp}{RGB}{197, 92, 212}
\definecolor{mygr}{HTML}{2C3338}
\definecolor{myred}{RGB}{127,0,0}
\definecolor{myyellow}{RGB}{169,121,69}
\definecolor{myexercisebg}{HTML}{F2FBF8}
\definecolor{myexercisefg}{HTML}{88D6D1}


%%%%%%%%%%%%%%%%%%%%%%%%%%%%
% TCOLORBOX SETUPS
%%%%%%%%%%%%%%%%%%%%%%%%%%%%

\setlength{\parindent}{1cm}
%================================
% THEOREM BOX
%================================

\tcbuselibrary{theorems,skins,hooks}
\newtcbtheorem[number within=section]{Theorem}{Theorem}
{%
	enhanced,
	breakable,
	colback = mytheorembg,
	frame hidden,
	boxrule = 0sp,
	borderline west = {2pt}{0pt}{mytheoremfr},
	sharp corners,
	detach title,
	before upper = \tcbtitle\par\smallskip,
	coltitle = mytheoremfr,
	fonttitle = \bfseries\sffamily,
	description font = \mdseries,
	separator sign none,
	segmentation style={solid, mytheoremfr},
}
{th}

\tcbuselibrary{theorems,skins,hooks}
\newtcbtheorem[number within=chapter]{theorem}{Theorem}
{%
	enhanced,
	breakable,
	colback = mytheorembg,
	frame hidden,
	boxrule = 0sp,
	borderline west = {2pt}{0pt}{mytheoremfr},
	sharp corners,
	detach title,
	before upper = \tcbtitle\par\smallskip,
	coltitle = mytheoremfr,
	fonttitle = \bfseries\sffamily,
	description font = \mdseries,
	separator sign none,
	segmentation style={solid, mytheoremfr},
}
{th}


\tcbuselibrary{theorems,skins,hooks}
\newtcolorbox{Theoremcon}
{%
	enhanced
	,breakable
	,colback = mytheorembg
	,frame hidden
	,boxrule = 0sp
	,borderline west = {2pt}{0pt}{mytheoremfr}
	,sharp corners
	,description font = \mdseries
	,separator sign none
}

%================================
% Corollery
%================================
\tcbuselibrary{theorems,skins,hooks}
\newtcbtheorem[number within=section]{Corollary}{Corollary}
{%
	enhanced
	,breakable
	,colback = myp!10
	,frame hidden
	,boxrule = 0sp
	,borderline west = {2pt}{0pt}{myp!85!black}
	,sharp corners
	,detach title
	,before upper = \tcbtitle\par\smallskip
	,coltitle = myp!85!black
	,fonttitle = \bfseries\sffamily
	,description font = \mdseries
	,separator sign none
	,segmentation style={solid, myp!85!black}
}
{th}
\tcbuselibrary{theorems,skins,hooks}
\newtcbtheorem[number within=chapter]{corollary}{Corollary}
{%
	enhanced
	,breakable
	,colback = myp!10
	,frame hidden
	,boxrule = 0sp
	,borderline west = {2pt}{0pt}{myp!85!black}
	,sharp corners
	,detach title
	,before upper = \tcbtitle\par\smallskip
	,coltitle = myp!85!black
	,fonttitle = \bfseries\sffamily
	,description font = \mdseries
	,separator sign none
	,segmentation style={solid, myp!85!black}
}
{th}


%================================
% LENMA
%================================

\tcbuselibrary{theorems,skins,hooks}
\newtcbtheorem[number within=section]{Lenma}{Lenma}
{%
	enhanced,
	breakable,
	colback = mylenmabg,
	frame hidden,
	boxrule = 0sp,
	borderline west = {2pt}{0pt}{mylenmafr},
	sharp corners,
	detach title,
	before upper = \tcbtitle\par\smallskip,
	coltitle = mylenmafr,
	fonttitle = \bfseries\sffamily,
	description font = \mdseries,
	separator sign none,
	segmentation style={solid, mylenmafr},
}
{th}

\tcbuselibrary{theorems,skins,hooks}
\newtcbtheorem[number within=chapter]{lenma}{Lenma}
{%
	enhanced,
	breakable,
	colback = mylenmabg,
	frame hidden,
	boxrule = 0sp,
	borderline west = {2pt}{0pt}{mylenmafr},
	sharp corners,
	detach title,
	before upper = \tcbtitle\par\smallskip,
	coltitle = mylenmafr,
	fonttitle = \bfseries\sffamily,
	description font = \mdseries,
	separator sign none,
	segmentation style={solid, mylenmafr},
}
{th}


%================================
% PROPOSITION
%================================

\tcbuselibrary{theorems,skins,hooks}
\newtcbtheorem[number within=section]{Prop}{Proposition}
{%
	enhanced,
	breakable,
	colback = mypropbg,
	frame hidden,
	boxrule = 0sp,
	borderline west = {2pt}{0pt}{mypropfr},
	sharp corners,
	detach title,
	before upper = \tcbtitle\par\smallskip,
	coltitle = mypropfr,
	fonttitle = \bfseries\sffamily,
	description font = \mdseries,
	separator sign none,
	segmentation style={solid, mypropfr},
}
{th}

\tcbuselibrary{theorems,skins,hooks}
\newtcbtheorem[number within=chapter]{prop}{Proposition}
{%
	enhanced,
	breakable,
	colback = mypropbg,
	frame hidden,
	boxrule = 0sp,
	borderline west = {2pt}{0pt}{mypropfr},
	sharp corners,
	detach title,
	before upper = \tcbtitle\par\smallskip,
	coltitle = mypropfr,
	fonttitle = \bfseries\sffamily,
	description font = \mdseries,
	separator sign none,
	segmentation style={solid, mypropfr},
}
{th}


%================================
% CLAIM
%================================

\tcbuselibrary{theorems,skins,hooks}
\newtcbtheorem[number within=section]{claim}{Claim}
{%
	enhanced
	,breakable
	,colback = myg!10
	,frame hidden
	,boxrule = 0sp
	,borderline west = {2pt}{0pt}{myg}
	,sharp corners
	,detach title
	,before upper = \tcbtitle\par\smallskip
	,coltitle = myg!85!black
	,fonttitle = \bfseries\sffamily
	,description font = \mdseries
	,separator sign none
	,segmentation style={solid, myg!85!black}
}
{th}



%================================
% Exercise
%================================

\tcbuselibrary{theorems,skins,hooks}
\newtcbtheorem[number within=section]{Exercise}{Exercise}
{%
	enhanced,
	breakable,
	colback = myexercisebg,
	frame hidden,
	boxrule = 0sp,
	borderline west = {2pt}{0pt}{myexercisefg},
	sharp corners,
	detach title,
	before upper = \tcbtitle\par\smallskip,
	coltitle = myexercisefg,
	fonttitle = \bfseries\sffamily,
	description font = \mdseries,
	separator sign none,
	segmentation style={solid, myexercisefg},
}
{th}

\tcbuselibrary{theorems,skins,hooks}
\newtcbtheorem[number within=chapter]{exercise}{Exercise}
{%
	enhanced,
	breakable,
	colback = myexercisebg,
	frame hidden,
	boxrule = 0sp,
	borderline west = {2pt}{0pt}{myexercisefg},
	sharp corners,
	detach title,
	before upper = \tcbtitle\par\smallskip,
	coltitle = myexercisefg,
	fonttitle = \bfseries\sffamily,
	description font = \mdseries,
	separator sign none,
	segmentation style={solid, myexercisefg},
}
{th}

%================================
% EXAMPLE BOX
%================================

\newtcbtheorem[number within=section]{Example}{Example}
{%
	colback = myexamplebg
	,breakable
	,colframe = myexamplefr
	,coltitle = myexampleti
	,boxrule = 1pt
	,sharp corners
	,detach title
	,before upper=\tcbtitle\par\smallskip
	,fonttitle = \bfseries
	,description font = \mdseries
	,separator sign none
	,description delimiters parenthesis
}
{ex}

\newtcbtheorem[number within=chapter]{example}{Example}
{%
	colback = myexamplebg
	,breakable
	,colframe = myexamplefr
	,coltitle = myexampleti
	,boxrule = 1pt
	,sharp corners
	,detach title
	,before upper=\tcbtitle\par\smallskip
	,fonttitle = \bfseries
	,description font = \mdseries
	,separator sign none
	,description delimiters parenthesis
}
{ex}

%================================
% DEFINITION BOX
%================================

\newtcbtheorem[number within=section]{Definition}{Definition}{enhanced,
	before skip=2mm,after skip=2mm, colback=red!5,colframe=red!80!black,boxrule=0.5mm,
	attach boxed title to top left={xshift=1cm,yshift*=1mm-\tcboxedtitleheight}, varwidth boxed title*=-3cm,
	boxed title style={frame code={
					\path[fill=tcbcolback]
					([yshift=-1mm,xshift=-1mm]frame.north west)
					arc[start angle=0,end angle=180,radius=1mm]
					([yshift=-1mm,xshift=1mm]frame.north east)
					arc[start angle=180,end angle=0,radius=1mm];
					\path[left color=tcbcolback!60!black,right color=tcbcolback!60!black,
						middle color=tcbcolback!80!black]
					([xshift=-2mm]frame.north west) -- ([xshift=2mm]frame.north east)
					[rounded corners=1mm]-- ([xshift=1mm,yshift=-1mm]frame.north east)
					-- (frame.south east) -- (frame.south west)
					-- ([xshift=-1mm,yshift=-1mm]frame.north west)
					[sharp corners]-- cycle;
				},interior engine=empty,
		},
	fonttitle=\bfseries,
	title={#2},#1}{def}
\newtcbtheorem[number within=chapter]{definition}{Definition}{enhanced,
	before skip=2mm,after skip=2mm, colback=red!5,colframe=red!80!black,boxrule=0.5mm,
	attach boxed title to top left={xshift=1cm,yshift*=1mm-\tcboxedtitleheight}, varwidth boxed title*=-3cm,
	boxed title style={frame code={
					\path[fill=tcbcolback]
					([yshift=-1mm,xshift=-1mm]frame.north west)
					arc[start angle=0,end angle=180,radius=1mm]
					([yshift=-1mm,xshift=1mm]frame.north east)
					arc[start angle=180,end angle=0,radius=1mm];
					\path[left color=tcbcolback!60!black,right color=tcbcolback!60!black,
						middle color=tcbcolback!80!black]
					([xshift=-2mm]frame.north west) -- ([xshift=2mm]frame.north east)
					[rounded corners=1mm]-- ([xshift=1mm,yshift=-1mm]frame.north east)
					-- (frame.south east) -- (frame.south west)
					-- ([xshift=-1mm,yshift=-1mm]frame.north west)
					[sharp corners]-- cycle;
				},interior engine=empty,
		},
	fonttitle=\bfseries,
	title={#2},#1}{def}



%================================
% Solution BOX
%================================

\makeatletter
\newtcbtheorem{question}{Question}{enhanced,
	breakable,
	colback=white,
	colframe=myb!80!black,
	attach boxed title to top left={yshift*=-\tcboxedtitleheight},
	fonttitle=\bfseries,
	title={#2},
	boxed title size=title,
	boxed title style={%
			sharp corners,
			rounded corners=northwest,
			colback=tcbcolframe,
			boxrule=0pt,
		},
	underlay boxed title={%
			\path[fill=tcbcolframe] (title.south west)--(title.south east)
			to[out=0, in=180] ([xshift=5mm]title.east)--
			(title.center-|frame.east)
			[rounded corners=\kvtcb@arc] |-
			(frame.north) -| cycle;
		},
	#1
}{def}
\makeatother

%================================
% SOLUTION BOX
%================================

\makeatletter
\newtcolorbox{solution}{enhanced,
	breakable,
	colback=white,
	colframe=myg!80!black,
	attach boxed title to top left={yshift*=-\tcboxedtitleheight},
	title=Solution,
	boxed title size=title,
	boxed title style={%
			sharp corners,
			rounded corners=northwest,
			colback=tcbcolframe,
			boxrule=0pt,
		},
	underlay boxed title={%
			\path[fill=tcbcolframe] (title.south west)--(title.south east)
			to[out=0, in=180] ([xshift=5mm]title.east)--
			(title.center-|frame.east)
			[rounded corners=\kvtcb@arc] |-
			(frame.north) -| cycle;
		},
}
\makeatother

%================================
% Question BOX
%================================

\makeatletter
\newtcbtheorem{qstion}{Question}{enhanced,
	breakable,
	colback=white,
	colframe=mygr,
	attach boxed title to top left={yshift*=-\tcboxedtitleheight},
	fonttitle=\bfseries,
	title={#2},
	boxed title size=title,
	boxed title style={%
			sharp corners,
			rounded corners=northwest,
			colback=tcbcolframe,
			boxrule=0pt,
		},
	underlay boxed title={%
			\path[fill=tcbcolframe] (title.south west)--(title.south east)
			to[out=0, in=180] ([xshift=5mm]title.east)--
			(title.center-|frame.east)
			[rounded corners=\kvtcb@arc] |-
			(frame.north) -| cycle;
		},
	#1
}{def}
\makeatother

\newtcbtheorem[number within=chapter]{wconc}{Wrong Concept}{
	breakable,
	enhanced,
	colback=white,
	colframe=myr,
	arc=0pt,
	outer arc=0pt,
	fonttitle=\bfseries\sffamily\large,
	colbacktitle=myr,
	attach boxed title to top left={},
	boxed title style={
			enhanced,
			skin=enhancedfirst jigsaw,
			arc=3pt,
			bottom=0pt,
			interior style={fill=myr}
		},
	#1
}{def}



%================================
% NOTE BOX
%================================

\usetikzlibrary{arrows,calc,shadows.blur}
\tcbuselibrary{skins}
\newtcolorbox{note}[1][]{%
	enhanced jigsaw,
	colback=gray!20!white,%
	colframe=gray!80!black,
	size=small,
	boxrule=1pt,
	title=\textbf{Note:-},
	halign title=flush center,
	coltitle=black,
	breakable,
	drop shadow=black!50!white,
	attach boxed title to top left={xshift=1cm,yshift=-\tcboxedtitleheight/2,yshifttext=-\tcboxedtitleheight/2},
	minipage boxed title=1.5cm,
	boxed title style={%
			colback=white,
			size=fbox,
			boxrule=1pt,
			boxsep=2pt,
			underlay={%
					\coordinate (dotA) at ($(interior.west) + (-0.5pt,0)$);
					\coordinate (dotB) at ($(interior.east) + (0.5pt,0)$);
					\begin{scope}
						\clip (interior.north west) rectangle ([xshift=3ex]interior.east);
						\filldraw [white, blur shadow={shadow opacity=60, shadow yshift=-.75ex}, rounded corners=2pt] (interior.north west) rectangle (interior.south east);
					\end{scope}
					\begin{scope}[gray!80!black]
						\fill (dotA) circle (2pt);
						\fill (dotB) circle (2pt);
					\end{scope}
				},
		},
	#1,
}

%%%%%%%%%%%%%%%%%%%%%%%%%%%%%%
% SELF MADE COMMANDS
%%%%%%%%%%%%%%%%%%%%%%%%%%%%%%


\newcommand{\thm}[2]{\begin{Theorem}{#1}{}#2\end{Theorem}}
\newcommand{\cor}[2]{\begin{Corollary}{#1}{}#2\end{Corollary}}
\newcommand{\mlenma}[2]{\begin{Lenma}{#1}{}#2\end{Lenma}}
\newcommand{\mprop}[2]{\begin{Prop}{#1}{}#2\end{Prop}}
\newcommand{\clm}[3]{\begin{claim}{#1}{#2}#3\end{claim}}
\newcommand{\wc}[2]{\begin{wconc}{#1}{}\setlength{\parindent}{1cm}#2\end{wconc}}
\newcommand{\thmcon}[1]{\begin{Theoremcon}{#1}\end{Theoremcon}}
\newcommand{\ex}[2]{\begin{Example}{#1}{}#2\end{Example}}
\newcommand{\dfn}[2]{\begin{Definition}[colbacktitle=red!75!black]{#1}{}#2\end{Definition}}
\newcommand{\dfnc}[2]{\begin{definition}[colbacktitle=red!75!black]{#1}{}#2\end{definition}}
\newcommand{\qs}[2]{\begin{question}{#1}{}#2\end{question}}
\newcommand{\pf}[2]{\begin{myproof}[#1]#2\end{myproof}}
\newcommand{\nt}[1]{\begin{note}#1\end{note}}

\newcommand*\circled[1]{\tikz[baseline=(char.base)]{
		\node[shape=circle,draw,inner sep=1pt] (char) {#1};}}
\newcommand\getcurrentref[1]{%
	\ifnumequal{\value{#1}}{0}
	{??}
	{\the\value{#1}}%
}
\newcommand{\getCurrentSectionNumber}{\getcurrentref{section}}
\newenvironment{myproof}[1][\proofname]{%
	\proof[\bfseries #1: ]%
}{\endproof}

\newcommand{\mclm}[2]{\begin{myclaim}[#1]#2\end{myclaim}}
\newenvironment{myclaim}[1][\claimname]{\proof[\bfseries #1: ]}{}

\newcounter{mylabelcounter}

\makeatletter
\newcommand{\setword}[2]{%
	\phantomsection
	#1\def\@currentlabel{\unexpanded{#1}}\label{#2}%
}
\makeatother




\tikzset{
	symbol/.style={
			draw=none,
			every to/.append style={
					edge node={node [sloped, allow upside down, auto=false]{$#1$}}}
		}
}


% deliminators
\DeclarePairedDelimiter{\abs}{\lvert}{\rvert}
\DeclarePairedDelimiter{\norm}{\lVert}{\rVert}

\DeclarePairedDelimiter{\ceil}{\lceil}{\rceil}
\DeclarePairedDelimiter{\floor}{\lfloor}{\rfloor}
\DeclarePairedDelimiter{\round}{\lfloor}{\rceil}

\newsavebox\diffdbox
\newcommand{\slantedromand}{{\mathpalette\makesl{d}}}
\newcommand{\makesl}[2]{%
\begingroup
\sbox{\diffdbox}{$\mathsurround=0pt#1\mathrm{#2}$}%
\pdfsave
\pdfsetmatrix{1 0 0.2 1}%
\rlap{\usebox{\diffdbox}}%
\pdfrestore
\hskip\wd\diffdbox
\endgroup
}
\newcommand{\dd}[1][]{\ensuremath{\mathop{}\!\ifstrempty{#1}{%
\slantedromand\@ifnextchar^{\hspace{0.2ex}}{\hspace{0.1ex}}}%
{\slantedromand\hspace{0.2ex}^{#1}}}}
\ProvideDocumentCommand\dv{o m g}{%
  \ensuremath{%
    \IfValueTF{#3}{%
      \IfNoValueTF{#1}{%
        \frac{\dd #2}{\dd #3}%
      }{%
        \frac{\dd^{#1} #2}{\dd #3^{#1}}%
      }%
    }{%
      \IfNoValueTF{#1}{%
        \frac{\dd}{\dd #2}%
      }{%
        \frac{\dd^{#1}}{\dd #2^{#1}}%
      }%
    }%
  }%
}
\providecommand*{\pdv}[3][]{\frac{\partial^{#1}#2}{\partial#3^{#1}}}
%  - others
\DeclareMathOperator{\Lap}{\mathcal{L}}
\DeclareMathOperator{\Var}{Var} % varience
\DeclareMathOperator{\Cov}{Cov} % covarience
\DeclareMathOperator{\E}{E} % expected

% Since the amsthm package isn't loaded

% I prefer the slanted \leq
\let\oldleq\leq % save them in case they're every wanted
\let\oldgeq\geq
\renewcommand{\leq}{\leqslant}
\renewcommand{\geq}{\geqslant}

% % redefine matrix env to allow for alignment, use r as default
% \renewcommand*\env@matrix[1][r]{\hskip -\arraycolsep
%     \let\@ifnextchar\new@ifnextchar
%     \array{*\c@MaxMatrixCols #1}}


%\usepackage{framed}
%\usepackage{titletoc}
%\usepackage{etoolbox}
%\usepackage{lmodern}


%\patchcmd{\tableofcontents}{\contentsname}{\sffamily\contentsname}{}{}

%\renewenvironment{leftbar}
%{\def\FrameCommand{\hspace{6em}%
%		{\color{myyellow}\vrule width 2pt depth 6pt}\hspace{1em}}%
%	\MakeFramed{\parshape 1 0cm \dimexpr\textwidth-6em\relax\FrameRestore}\vskip2pt%
%}
%{\endMakeFramed}

%\titlecontents{chapter}
%[0em]{\vspace*{2\baselineskip}}
%{\parbox{4.5em}{%
%		\hfill\Huge\sffamily\bfseries\color{myred}\thecontentspage}%
%	\vspace*{-2.3\baselineskip}\leftbar\textsc{\small\chaptername~\thecontentslabel}\\\sffamily}
%{}{\endleftbar}
%\titlecontents{section}
%[8.4em]
%{\sffamily\contentslabel{3em}}{}{}
%{\hspace{0.5em}\nobreak\itshape\color{myred}\contentspage}
%\titlecontents{subsection}
%[8.4em]
%{\sffamily\contentslabel{3em}}{}{}  
%{\hspace{0.5em}\nobreak\itshape\color{myred}\contentspage}



%%%%%%%%%%%%%%%%%%%%%%%%%%%%%%%%%%%%%%%%%%%
% TABLE OF CONTENTS
%%%%%%%%%%%%%%%%%%%%%%%%%%%%%%%%%%%%%%%%%%%

\usepackage{tikz}
\definecolor{doc}{RGB}{0,60,110}
\usepackage{titletoc}
\contentsmargin{0cm}
\titlecontents{chapter}[3.7pc]
{\addvspace{30pt}%
	\begin{tikzpicture}[remember picture, overlay]%
		\draw[fill=doc!60,draw=doc!60] (-7,-.1) rectangle (-0.9,.5);%
		\pgftext[left,x=-3.5cm,y=0.2cm]{\color{white}\Large\sc\bfseries Chapter\ \thecontentslabel};%
	\end{tikzpicture}\color{doc!60}\large\sc\bfseries}%
{}
{}
{\;\titlerule\;\large\sc\bfseries Page \thecontentspage
	\begin{tikzpicture}[remember picture, overlay]
		\draw[fill=doc!60,draw=doc!60] (2pt,0) rectangle (4,0.1pt);
	\end{tikzpicture}}%
\titlecontents{section}[3.7pc]
{\addvspace{2pt}}
{\contentslabel[\thecontentslabel]{2pc}}
{}
{\hfill\small \thecontentspage}
[]
\titlecontents*{subsection}[3.7pc]
{\addvspace{-1pt}\small}
{}
{}
{\ --- \small\thecontentspage}
[ \textbullet\ ][]

\makeatletter
\renewcommand{\tableofcontents}{%
	\chapter*{%
	  \vspace*{-20\p@}%
	  \begin{tikzpicture}[remember picture, overlay]%
		  \pgftext[right,x=15cm,y=0.2cm]{\color{doc!60}\Huge\sc\bfseries \contentsname};%
		  \draw[fill=doc!60,draw=doc!60] (13,-.75) rectangle (20,1);%
		  \clip (13,-.75) rectangle (20,1);
		  \pgftext[right,x=15cm,y=0.2cm]{\color{white}\Huge\sc\bfseries \contentsname};%
	  \end{tikzpicture}}%
	\@starttoc{toc}}
\makeatother

%From M275 "Topology" at SJSU
\newcommand{\id}{\mathrm{id}}
\newcommand{\taking}[1]{\xrightarrow{#1}}
\newcommand{\inv}{^{-1}}

%From M170 "Introduction to Graph Theory" at SJSU
\DeclareMathOperator{\diam}{diam}
\DeclareMathOperator{\ord}{ord}
\newcommand{\defeq}{\overset{\mathrm{def}}{=}}

%From the USAMO .tex files
\newcommand{\ts}{\textsuperscript}
\newcommand{\dg}{^\circ}
\newcommand{\ii}{\item}

% % From Math 55 and Math 145 at Harvard
% \newenvironment{subproof}[1][Proof]{%
% \begin{proof}[#1] \renewcommand{\qedsymbol}{$\blacksquare$}}%
% {\end{proof}}

\newcommand{\liff}{\leftrightarrow}
\newcommand{\lthen}{\rightarrow}
\newcommand{\opname}{\operatorname}
\newcommand{\surjto}{\twoheadrightarrow}
\newcommand{\injto}{\hookrightarrow}
\newcommand{\On}{\mathrm{On}} % ordinals
\DeclareMathOperator{\img}{im} % Image
\DeclareMathOperator{\Img}{Im} % Image
\DeclareMathOperator{\coker}{coker} % Cokernel
\DeclareMathOperator{\Coker}{Coker} % Cokernel
\DeclareMathOperator{\Ker}{Ker} % Kernel
\DeclareMathOperator{\rank}{rank}
\DeclareMathOperator{\Spec}{Spec} % spectrum
\DeclareMathOperator{\Tr}{Tr} % trace
\DeclareMathOperator{\pr}{pr} % projection
\DeclareMathOperator{\ext}{ext} % extension
\DeclareMathOperator{\pred}{pred} % predecessor
\DeclareMathOperator{\dom}{dom} % domain
\DeclareMathOperator{\ran}{ran} % range
\DeclareMathOperator{\Hom}{Hom} % homomorphism
\DeclareMathOperator{\Mor}{Mor} % morphisms
\DeclareMathOperator{\End}{End} % endomorphism

\newcommand{\eps}{\epsilon}
\newcommand{\veps}{\varepsilon}
\newcommand{\ol}{\overline}
\newcommand{\ul}{\underline}
\newcommand{\wt}{\widetilde}
\newcommand{\wh}{\widehat}
\newcommand{\vocab}[1]{\textbf{\color{blue} #1}}
\providecommand{\half}{\frac{1}{2}}
\newcommand{\dang}{\measuredangle} %% Directed angle
\newcommand{\ray}[1]{\overrightarrow{#1}}
\newcommand{\seg}[1]{\overline{#1}}
\newcommand{\arc}[1]{\wideparen{#1}}
\DeclareMathOperator{\cis}{cis}
\DeclareMathOperator*{\lcm}{lcm}
\DeclareMathOperator*{\argmin}{arg min}
\DeclareMathOperator*{\argmax}{arg max}
\newcommand{\cycsum}{\sum_{\mathrm{cyc}}}
\newcommand{\symsum}{\sum_{\mathrm{sym}}}
\newcommand{\cycprod}{\prod_{\mathrm{cyc}}}
\newcommand{\symprod}{\prod_{\mathrm{sym}}}
\newcommand{\Qed}{\begin{flushright}\qed\end{flushright}}
\newcommand{\parinn}{\setlength{\parindent}{1cm}}
\newcommand{\parinf}{\setlength{\parindent}{0cm}}
% \newcommand{\norm}{\|\cdot\|}
\newcommand{\inorm}{\norm_{\infty}}
\newcommand{\opensets}{\{V_{\alpha}\}_{\alpha\in I}}
\newcommand{\oset}{V_{\alpha}}
\newcommand{\opset}[1]{V_{\alpha_{#1}}}
\newcommand{\lub}{\text{lub}}
\newcommand{\del}[2]{\frac{\partial #1}{\partial #2}}
\newcommand{\Del}[3]{\frac{\partial^{#1} #2}{\partial^{#1} #3}}
\newcommand{\deld}[2]{\dfrac{\partial #1}{\partial #2}}
\newcommand{\Deld}[3]{\dfrac{\partial^{#1} #2}{\partial^{#1} #3}}
\newcommand{\lm}{\lambda}
\newcommand{\uin}{\mathbin{\rotatebox[origin=c]{90}{$\in$}}}
\newcommand{\usubset}{\mathbin{\rotatebox[origin=c]{90}{$\subset$}}}
\newcommand{\lt}{\left}
\newcommand{\rt}{\right}
\newcommand{\bs}[1]{\boldsymbol{#1}}
\newcommand{\exs}{\exists}
\newcommand{\st}{\strut}
\newcommand{\dps}[1]{\displaystyle{#1}}

\newcommand{\sol}{\setlength{\parindent}{0cm}\textbf{\textit{Solution:}}\setlength{\parindent}{1cm} }
\newcommand{\solve}[1]{\setlength{\parindent}{0cm}\textbf{\textit{Solution: }}\setlength{\parindent}{1cm}#1 \Qed}
% Things Lie
\newcommand{\kb}{\mathfrak b}
\newcommand{\kg}{\mathfrak g}
\newcommand{\kh}{\mathfrak h}
\newcommand{\kn}{\mathfrak n}
\newcommand{\ku}{\mathfrak u}
\newcommand{\kz}{\mathfrak z}
\DeclareMathOperator{\Ext}{Ext} % Ext functor
\DeclareMathOperator{\Tor}{Tor} % Tor functor
\newcommand{\gl}{\opname{\mathfrak{gl}}} % frak gl group
\renewcommand{\sl}{\opname{\mathfrak{sl}}} % frak sl group chktex 6

% More script letters etc.
\newcommand{\SA}{\mathcal A}
\newcommand{\SB}{\mathcal B}
\newcommand{\SC}{\mathcal C}
\newcommand{\SF}{\mathcal F}
\newcommand{\SG}{\mathcal G}
\newcommand{\SH}{\mathcal H}
\newcommand{\OO}{\mathcal O}

\newcommand{\SCA}{\mathscr A}
\newcommand{\SCB}{\mathscr B}
\newcommand{\SCC}{\mathscr C}
\newcommand{\SCD}{\mathscr D}
\newcommand{\SCE}{\mathscr E}
\newcommand{\SCF}{\mathscr F}
\newcommand{\SCG}{\mathscr G}
\newcommand{\SCH}{\mathscr H}

% Mathfrak primes
\newcommand{\km}{\mathfrak m}
\newcommand{\kp}{\mathfrak p}
\newcommand{\kq}{\mathfrak q}

% number sets
\newcommand{\RR}[1][]{\ensuremath{\ifstrempty{#1}{\mathbb{R}}{\mathbb{R}^{#1}}}}
\newcommand{\NN}[1][]{\ensuremath{\ifstrempty{#1}{\mathbb{N}}{\mathbb{N}^{#1}}}}
\newcommand{\ZZ}[1][]{\ensuremath{\ifstrempty{#1}{\mathbb{Z}}{\mathbb{Z}^{#1}}}}
\newcommand{\QQ}[1][]{\ensuremath{\ifstrempty{#1}{\mathbb{Q}}{\mathbb{Q}^{#1}}}}
\newcommand{\CC}[1][]{\ensuremath{\ifstrempty{#1}{\mathbb{C}}{\mathbb{C}^{#1}}}}
\newcommand{\PP}[1][]{\ensuremath{\ifstrempty{#1}{\mathbb{P}}{\mathbb{P}^{#1}}}}
\newcommand{\HH}[1][]{\ensuremath{\ifstrempty{#1}{\mathbb{H}}{\mathbb{H}^{#1}}}}
\newcommand{\FF}[1][]{\ensuremath{\ifstrempty{#1}{\mathbb{F}}{\mathbb{F}^{#1}}}}
% expected value
\newcommand{\EE}{\ensuremath{\mathbb{E}}}
\newcommand{\charin}{\text{ char }}
\DeclareMathOperator{\sign}{sign}
\DeclareMathOperator{\Aut}{Aut}
\DeclareMathOperator{\Inn}{Inn}
\DeclareMathOperator{\Syl}{Syl}
\DeclareMathOperator{\Gal}{Gal}
\DeclareMathOperator{\GL}{GL} % General linear group
\DeclareMathOperator{\SL}{SL} % Special linear group

%---------------------------------------
% BlackBoard Math Fonts :-
%---------------------------------------

%Captital Letters
\newcommand{\bbA}{\mathbb{A}}	\newcommand{\bbB}{\mathbb{B}}
\newcommand{\bbC}{\mathbb{C}}	\newcommand{\bbD}{\mathbb{D}}
\newcommand{\bbE}{\mathbb{E}}	\newcommand{\bbF}{\mathbb{F}}
\newcommand{\bbG}{\mathbb{G}}	\newcommand{\bbH}{\mathbb{H}}
\newcommand{\bbI}{\mathbb{I}}	\newcommand{\bbJ}{\mathbb{J}}
\newcommand{\bbK}{\mathbb{K}}	\newcommand{\bbL}{\mathbb{L}}
\newcommand{\bbM}{\mathbb{M}}	\newcommand{\bbN}{\mathbb{N}}
\newcommand{\bbO}{\mathbb{O}}	\newcommand{\bbP}{\mathbb{P}}
\newcommand{\bbQ}{\mathbb{Q}}	\newcommand{\bbR}{\mathbb{R}}
\newcommand{\bbS}{\mathbb{S}}	\newcommand{\bbT}{\mathbb{T}}
\newcommand{\bbU}{\mathbb{U}}	\newcommand{\bbV}{\mathbb{V}}
\newcommand{\bbW}{\mathbb{W}}	\newcommand{\bbX}{\mathbb{X}}
\newcommand{\bbY}{\mathbb{Y}}	\newcommand{\bbZ}{\mathbb{Z}}

%---------------------------------------
% MathCal Fonts :-
%---------------------------------------

%Captital Letters
\newcommand{\mcA}{\mathcal{A}}	\newcommand{\mcB}{\mathcal{B}}
\newcommand{\mcC}{\mathcal{C}}	\newcommand{\mcD}{\mathcal{D}}
\newcommand{\mcE}{\mathcal{E}}	\newcommand{\mcF}{\mathcal{F}}
\newcommand{\mcG}{\mathcal{G}}	\newcommand{\mcH}{\mathcal{H}}
\newcommand{\mcI}{\mathcal{I}}	\newcommand{\mcJ}{\mathcal{J}}
\newcommand{\mcK}{\mathcal{K}}	\newcommand{\mcL}{\mathcal{L}}
\newcommand{\mcM}{\mathcal{M}}	\newcommand{\mcN}{\mathcal{N}}
\newcommand{\mcO}{\mathcal{O}}	\newcommand{\mcP}{\mathcal{P}}
\newcommand{\mcQ}{\mathcal{Q}}	\newcommand{\mcR}{\mathcal{R}}
\newcommand{\mcS}{\mathcal{S}}	\newcommand{\mcT}{\mathcal{T}}
\newcommand{\mcU}{\mathcal{U}}	\newcommand{\mcV}{\mathcal{V}}
\newcommand{\mcW}{\mathcal{W}}	\newcommand{\mcX}{\mathcal{X}}
\newcommand{\mcY}{\mathcal{Y}}	\newcommand{\mcZ}{\mathcal{Z}}


%---------------------------------------
% Bold Math Fonts :-
%---------------------------------------

%Captital Letters
\newcommand{\bmA}{\boldsymbol{A}}	\newcommand{\bmB}{\boldsymbol{B}}
\newcommand{\bmC}{\boldsymbol{C}}	\newcommand{\bmD}{\boldsymbol{D}}
\newcommand{\bmE}{\boldsymbol{E}}	\newcommand{\bmF}{\boldsymbol{F}}
\newcommand{\bmG}{\boldsymbol{G}}	\newcommand{\bmH}{\boldsymbol{H}}
\newcommand{\bmI}{\boldsymbol{I}}	\newcommand{\bmJ}{\boldsymbol{J}}
\newcommand{\bmK}{\boldsymbol{K}}	\newcommand{\bmL}{\boldsymbol{L}}
\newcommand{\bmM}{\boldsymbol{M}}	\newcommand{\bmN}{\boldsymbol{N}}
\newcommand{\bmO}{\boldsymbol{O}}	\newcommand{\bmP}{\boldsymbol{P}}
\newcommand{\bmQ}{\boldsymbol{Q}}	\newcommand{\bmR}{\boldsymbol{R}}
\newcommand{\bmS}{\boldsymbol{S}}	\newcommand{\bmT}{\boldsymbol{T}}
\newcommand{\bmU}{\boldsymbol{U}}	\newcommand{\bmV}{\boldsymbol{V}}
\newcommand{\bmW}{\boldsymbol{W}}	\newcommand{\bmX}{\boldsymbol{X}}
\newcommand{\bmY}{\boldsymbol{Y}}	\newcommand{\bmZ}{\boldsymbol{Z}}
%Small Letters
\newcommand{\bma}{\boldsymbol{a}}	\newcommand{\bmb}{\boldsymbol{b}}
\newcommand{\bmc}{\boldsymbol{c}}	\newcommand{\bmd}{\boldsymbol{d}}
\newcommand{\bme}{\boldsymbol{e}}	\newcommand{\bmf}{\boldsymbol{f}}
\newcommand{\bmg}{\boldsymbol{g}}	\newcommand{\bmh}{\boldsymbol{h}}
\newcommand{\bmi}{\boldsymbol{i}}	\newcommand{\bmj}{\boldsymbol{j}}
\newcommand{\bmk}{\boldsymbol{k}}	\newcommand{\bml}{\boldsymbol{l}}
\newcommand{\bmm}{\boldsymbol{m}}	\newcommand{\bmn}{\boldsymbol{n}}
\newcommand{\bmo}{\boldsymbol{o}}	\newcommand{\bmp}{\boldsymbol{p}}
\newcommand{\bmq}{\boldsymbol{q}}	\newcommand{\bmr}{\boldsymbol{r}}
\newcommand{\bms}{\boldsymbol{s}}	\newcommand{\bmt}{\boldsymbol{t}}
\newcommand{\bmu}{\boldsymbol{u}}	\newcommand{\bmv}{\boldsymbol{v}}
\newcommand{\bmw}{\boldsymbol{w}}	\newcommand{\bmx}{\boldsymbol{x}}
\newcommand{\bmy}{\boldsymbol{y}}	\newcommand{\bmz}{\boldsymbol{z}}

%---------------------------------------
% Scr Math Fonts :-
%---------------------------------------

\newcommand{\sA}{{\mathscr{A}}}   \newcommand{\sB}{{\mathscr{B}}}
\newcommand{\sC}{{\mathscr{C}}}   \newcommand{\sD}{{\mathscr{D}}}
\newcommand{\sE}{{\mathscr{E}}}   \newcommand{\sF}{{\mathscr{F}}}
\newcommand{\sG}{{\mathscr{G}}}   \newcommand{\sH}{{\mathscr{H}}}
\newcommand{\sI}{{\mathscr{I}}}   \newcommand{\sJ}{{\mathscr{J}}}
\newcommand{\sK}{{\mathscr{K}}}   \newcommand{\sL}{{\mathscr{L}}}
\newcommand{\sM}{{\mathscr{M}}}   \newcommand{\sN}{{\mathscr{N}}}
\newcommand{\sO}{{\mathscr{O}}}   \newcommand{\sP}{{\mathscr{P}}}
\newcommand{\sQ}{{\mathscr{Q}}}   \newcommand{\sR}{{\mathscr{R}}}
\newcommand{\sS}{{\mathscr{S}}}   \newcommand{\sT}{{\mathscr{T}}}
\newcommand{\sU}{{\mathscr{U}}}   \newcommand{\sV}{{\mathscr{V}}}
\newcommand{\sW}{{\mathscr{W}}}   \newcommand{\sX}{{\mathscr{X}}}
\newcommand{\sY}{{\mathscr{Y}}}   \newcommand{\sZ}{{\mathscr{Z}}}


%---------------------------------------
% Math Fraktur Font
%---------------------------------------

%Captital Letters
\newcommand{\mfA}{\mathfrak{A}}	\newcommand{\mfB}{\mathfrak{B}}
\newcommand{\mfC}{\mathfrak{C}}	\newcommand{\mfD}{\mathfrak{D}}
\newcommand{\mfE}{\mathfrak{E}}	\newcommand{\mfF}{\mathfrak{F}}
\newcommand{\mfG}{\mathfrak{G}}	\newcommand{\mfH}{\mathfrak{H}}
\newcommand{\mfI}{\mathfrak{I}}	\newcommand{\mfJ}{\mathfrak{J}}
\newcommand{\mfK}{\mathfrak{K}}	\newcommand{\mfL}{\mathfrak{L}}
\newcommand{\mfM}{\mathfrak{M}}	\newcommand{\mfN}{\mathfrak{N}}
\newcommand{\mfO}{\mathfrak{O}}	\newcommand{\mfP}{\mathfrak{P}}
\newcommand{\mfQ}{\mathfrak{Q}}	\newcommand{\mfR}{\mathfrak{R}}
\newcommand{\mfS}{\mathfrak{S}}	\newcommand{\mfT}{\mathfrak{T}}
\newcommand{\mfU}{\mathfrak{U}}	\newcommand{\mfV}{\mathfrak{V}}
\newcommand{\mfW}{\mathfrak{W}}	\newcommand{\mfX}{\mathfrak{X}}
\newcommand{\mfY}{\mathfrak{Y}}	\newcommand{\mfZ}{\mathfrak{Z}}
%Small Letters
\newcommand{\mfa}{\mathfrak{a}}	\newcommand{\mfb}{\mathfrak{b}}
\newcommand{\mfc}{\mathfrak{c}}	\newcommand{\mfd}{\mathfrak{d}}
\newcommand{\mfe}{\mathfrak{e}}	\newcommand{\mff}{\mathfrak{f}}
\newcommand{\mfg}{\mathfrak{g}}	\newcommand{\mfh}{\mathfrak{h}}
\newcommand{\mfi}{\mathfrak{i}}	\newcommand{\mfj}{\mathfrak{j}}
\newcommand{\mfk}{\mathfrak{k}}	\newcommand{\mfl}{\mathfrak{l}}
\newcommand{\mfm}{\mathfrak{m}}	\newcommand{\mfn}{\mathfrak{n}}
\newcommand{\mfo}{\mathfrak{o}}	\newcommand{\mfp}{\mathfrak{p}}
\newcommand{\mfq}{\mathfrak{q}}	\newcommand{\mfr}{\mathfrak{r}}
\newcommand{\mfs}{\mathfrak{s}}	\newcommand{\mft}{\mathfrak{t}}
\newcommand{\mfu}{\mathfrak{u}}	\newcommand{\mfv}{\mathfrak{v}}
\newcommand{\mfw}{\mathfrak{w}}	\newcommand{\mfx}{\mathfrak{x}}
\newcommand{\mfy}{\mathfrak{y}}	\newcommand{\mfz}{\mathfrak{z}}

\lstset{
    language=Python,
    tabsize=4,
    showspaces=false,
    showstringspaces=false,
    keepspaces=true,
    commentstyle=\color{green!50!black},
    keywordstyle=\color{blue},
    stringstyle=\color{red}
}

\begin{document}

\pagebreak

\setcounter{chapter}{1}
\setcounter{section}{0}
\section{Python}


\qs{}{
	Explain Installation of Python.
}
\sol{}

\begin{verbatim}
	Windows:

		Download Python Installer: Visit the official Python website, 
		download the installer, and choose the appropriate version (32-bit or 64-bit).

		Run Installer: Double-click the installer file and follow the prompts. 
		Make sure to check the box that says "Add Python to PATH".

		Complete Installation: Follow the installation wizard's instructions. 
		Python will be installed in C:\PythonXX\ (where XX is the version number).

		Verify Installation: Open a command prompt, type python --version, and press Enter.
		 You should see the installed Python version.

	macOS:

		Install Homebrew (optional): If you prefer Homebrew, 
		open Terminal and run the Homebrew installation command.

		Install Python: Use Homebrew (brew install python) or 
		download the macOS installer from python.org and run it.

		Verify Installation: Open Terminal and type 
		python3 --version to verify the installation.

	Linux (Ubuntu/Debian):

		Update Package Lists: Open Terminal and run sudo apt update.

		Install Python: Run sudo apt install python3 in Terminal to install Python 3.

		Verify Installation: After installation, type python3 --version in Terminal to verify.

\end{verbatim}

\newpage

\qs{}{To print different types of data 
	types with multiline and single line comments.}

\sol{}

\begin{lstlisting}[language=Python]

	# Single-line comment: Printing different data types
	print("Data types:")

	# Integer data type
	num = 10
	print("Integer:", num)

	# Float data type
	float_num = 3.14
	print("Float:", float_num)

	# String data type
	string = "Hello, World!"
	print("String:", string)

	# Boolean data type
	bool_value = True
	print("Boolean:", bool_value)

	"""
	Multiline comment:
	List data type
	"""
	my_list = [1, 2, 3, 4, 5]
	print("List:", my_list)

	"""
	Multiline comment:
	Tuple data type
	"""
	my_tuple = (1, 2, 3, 4, 5)
	print("Tuple:", my_tuple)

	"""
	Multiline comment:
	Dictionary data type
	"""
	my_dict = {'a': 1, 'b': 2, 'c': 3}
	print("Dictionary:", my_dict)

	"""
	Multiline comment:
	Set data type
	"""
	my_set = {1, 2, 3, 4, 5}
	print("Set:", my_set)

	print("\n")
	print("Rishabh Rathore")
	print("0827CI211155")
\end{lstlisting}

\begin{verbatim}

	Output:

	Data types:
	Integer: 10
	Float: 3.14
	String: Hello, World!
	Boolean: True
	List: [1, 2, 3, 4, 5]
	Tuple: (1, 2, 3, 4, 5)
	Dictionary: {'a': 1, 'b': 2, 'c': 3}
	Set: {1, 2, 3, 4, 5}
	
	
	Rishabh Rathore
	0827CI211155
\end{verbatim}

\newpage

\qs{}{To print the largest number of three numbers using if else and elif.}

\sol{}

\begin{lstlisting}[language=Python]

	# Three numbers
	num1 = 10
	num2 = 20
	num3 = 15

	# Check which number is largest
	if num1 >= num2 and num1 >= num3:
		largest_number = num1
	elif num2 >= num1 and num2 >= num3:
		largest_number = num2
	else:
		largest_number = num3

	# Print the result
	print("The largest number among", num1, ",", 
		num2, ", and", num3, "is:", largest_number)
	print("\n")
	print("Rishabh Rathore")
	print("0827CI211155")
\end{lstlisting}

\begin{verbatim}
	Output:

	The largest number among 10 , 20 , and 15 is: 20


	Rishabh Rathore
	0827CI211155
\end{verbatim}
\newpage


\qs{}{Write a program to print number I-IO using a while loop.}
\sol{}
\begin{lstlisting}[language=Python]
	# Initialize the starting number
	number = 1

	# Print numbers from 1 to 10 using a while loop
	while number <= 10:
		print(number)
		number += 1

	print("\n")
	print("Rishabh Rathore")
	print("0827CI211155")

\end{lstlisting}

\begin{verbatim}
	Output:

	1
	2
	3
	4
	5
	6
	7
	8
	9
	10
	
	
	Rishabh Rathore
	0827CI211155
\end{verbatim}
\newpage


\qs{}{
Write a program to check prime numbers between 1-100 using a for loop.
}
\sol{}
\begin{lstlisting}[language=Python]

	# Iterate through numbers from 1 to 100
	for num in range(1, 101):
		# Check if the number is greater than 1
		if num > 1:
			# Check for factors
			for i in range(2, int(num ** 0.5) + 1):
				if (num % i) == 0:
					break
			else:
				print(num, "is a prime number.")

	print("\n")
	print("Rishabh Rathore")
	print("0827CI211155")
\end{lstlisting}

\begin{verbatim}
	Output:

	2 is a prime number.
	3 is a prime number.
	5 is a prime number.
	7 is a prime number.
	11 is a prime number.
	13 is a prime number.
	17 is a prime number.
	19 is a prime number.
	23 is a prime number.
	29 is a prime number.
	31 is a prime number.
	37 is a prime number.
	41 is a prime number.
	43 is a prime number.
	47 is a prime number.
	53 is a prime number.
	59 is a prime number.
	61 is a prime number.
	67 is a prime number.
	71 is a prime number.
	73 is a prime number.
	79 is a prime number.
	83 is a prime number.
	89 is a prime number.
	97 is a prime number.
	
	
	Rishabh Rathore
	0827CI211155
\end{verbatim}
\newpage

\qs{}{Write a program to concatenate two strings using + operator}
\sol{}
\begin{lstlisting}[language=Python]
	# Two strings to concatenate
	string1 = "Hello world"
	string2 = ".This is always constant"

	# Concatenate the strings using the + operator
	concatenated_string = string1 + " " + string2

	# Print the concatenated string
	print("Concatenated string:", concatenated_string)

	print("\n")
	print("Rishabh Rathore")
	print("0827CI211155")
\end{lstlisting}

\begin{verbatim}
	Output:

	Concatenated string: Hello world .This is always constant


	Rishabh Rathore
	0827CI211155
\end{verbatim}
\newpage


\qs{}{
Write a program to reverse a string using slicing
}
\sol{}
\begin{lstlisting}[language=Python]

	# Input string
	input_string = "Hello, World!"

	# Reverse the string using slicing
	reversed_string = input_string[::-1]

	# Print the reversed string
	print("Original String:", input_string)
	print("Reversed String:", reversed_string)
	print("\n")
	print("Rishabh Rathore")
	print("0827CI211155")
\end{lstlisting}

\begin{verbatim}
	Output:

	Original String: Hello, World!
	Reversed String: !dlroW ,olleH
	
	
	Rishabh Rathore
	0827CI211155
\end{verbatim}
\newpage

\qs{}{
Write a program to perform different methods of string like: len(atleast 5)
}
\sol{}
\begin{lstlisting}[language=Python]
	# Input string
	input_string = "Hello, World!"

	# Length of the string
	length = len(input_string)
	print("Length of the string:", length)

	# Convert string to uppercase
	uppercase_string = input_string.upper()
	print("Uppercase string:", uppercase_string)

	# Convert string to lowercase
	lowercase_string = input_string.lower()
	print("Lowercase string:", lowercase_string)

	# Count occurrences of a substring
	substring = "o"
	count = input_string.count(substring)
	print("Number of occurrences of 'o':", count)

	# Replace substring
	old_substring = "World"
	new_substring = "Python"
	replaced_string = input_string.replace(old_substring, new_substring)
	print("String after replacement:", replaced_string)

	# Check if string starts with a substring
	substring_to_check = "Hello"
	starts_with = input_string.startswith(substring_to_check)
	print(f"String starts with '{substring_to_check}':", starts_with)

	# Check if string ends with a substring
	substring_to_check = "!"
	ends_with = input_string.endswith(substring_to_check)
	print(f"String ends with '{substring_to_check}':", ends_with)

	print("\n")
	print("Rishabh Rathore")
	print("0827CI211155")
\end{lstlisting}

\begin{verbatim}
	Output:

	Length of the string: 13
	Uppercase string: HELLO, WORLD!
	Lowercase string: hello, world!
	Number of occurrences of 'o': 2
	String after replacement: Hello, Python!
	String starts with 'Hello': True
	String ends with '!': True
	
	
	Rishabh Rathore
	0827CI211155
\end{verbatim}
\newpage


\qs{}{
Write a program to traverse all the characters of a string using a for loop.
}
\sol{}
\begin{lstlisting}[language=Python]
	# Input string
	input_string = "Hello, World!"

	# Traverse all characters of the string using a for loop
	print("Traversing all characters of the string using a for loop:")
	for char in input_string:
		print(char)

	print("\n")
	print("Rishabh Rathore")
	print("0827CI211155")
\end{lstlisting}

\begin{verbatim}
	Output:

	Traversing all characters of the string using a for loop:
	H
	e
	l
	l
	o
	,
	 
	W
	o
	r
	l
	d
	!
	
	
	Rishabh Rathore
	0827CI211155
\end{verbatim}
\newpage


\qs{}{
 Write a program to Print abecederian series
}
\sol{}
\begin{lstlisting}[language=Python]

	# Function to check if a word is abecedarian
	def is_abecedarian(word):
		"""
		Function to check if a word is abecedarian.

		Parameters:
		word (str): The word to check.

		Returns:
		bool: True if the word is abecedarian, False otherwise.
		"""
		# Iterate through each character in the word
		for i in range(len(word) - 1):
			# Check if the current character is greater than the next character
			if word[i] > word[i + 1]:
				return False
		return True

	# Main program to print abecedarian series
	def print_abecedarian_series():
		"""
		Function to print abecedarian series.
		"""
		# Starting letter
		start_letter = 'a'

		# Loop to generate and print abecedarian words
		while start_letter <= 'z':
			print(start_letter)
			start_letter = chr(ord(start_letter) + 1)

	# Print the abecedarian series
	print("Abecedarian series:")
	print_abecedarian_series()
	print("\n")
	print("Rishabh Rathore")
	print("0827CI211155")
\end{lstlisting}

\begin{verbatim}
	Output:

	Abecedarian series:
	a
	b
	c
	d
	e
	f
	g
	h
	i
	j
	k
	l
	m
	n
	o
	p
	q
	r
	s
	t
	u
	v
	w
	x
	...
	
	
	Rishabh Rathore
	0827CI211155
\end{verbatim}
\newpage


\qs{}{
Write a program to check whether a string is present in another string or not.
}
\sol{}
\begin{lstlisting}[language=Python]

	def check_substring(main_string, substring):
		"""
		Function to check whether a substring is present in a main string.

		Parameters:
		main_string (str): The main string.
		substring (str): The substring to check.

		Returns:
		bool: True if substring is present in main string, False otherwise.
		"""
		return substring in main_string

	# Example usage:
	main_string = "Hello, World!"
	substring1 = "Hello"
	substring2 = "Python"

	# Check if substrings are present in the main string
	print(f"'\{substring1\}' is present in '\{main_string\}':", check_substring(main_string, substring1))
	print(f"'\{substring2\}' is present in '\{main_string\}':", check_substring(main_string, substring2))
	print("\n")
	print("Rishabh Rathore")
	print("0827CI211155")

\end{lstlisting}

\begin{verbatim}
	Output:
	
	'Hello' is present in 'Hello, World!': True
	'Python' is present in 'Hello, World!': False
	
	
	Rishabh Rathore
	0827CI211155
\end{verbatim}
\newpage


\qs{}{
 Write a program to print \\
 A \\
 ABC \\
 ABCD \\
}
\sol{}
\begin{lstlisting}[language=Python]

	# Number of rows for the pattern
	num_rows = 4

	# Outer loop to iterate over each rowi
	for i in range(1, num_rows + 1):
		# Inner loop to print characters from 'A' to 'A + i - 1'
		for j in range(ord('A'), ord('A') + i):
			print(chr(j), end="")
		print()  # Move to the next line after printing each row
	print("\n")
	print("Rishabh Rathore")
	print("0827CI211155")
\end{lstlisting}

\begin{verbatim}
	Output:

	A
	AB
	ABC
	ABCD
	
	
	Rishabh Rathore
	0827CI211155

\end{verbatim}
\newpage


\qs{}{
Write a program to Create a list with different data types.
}
\sol{}
\begin{lstlisting}[language=Python]
# Create a list with different data types
my_list = [1, "Hello", 3.14, True, [1, 2, 3]]

# Print the list
print("List with different data types:", my_list)
print("\n")
print("Rishabh Rathore")
print("0827CI211155")
\end{lstlisting}

\begin{verbatim}
	Output:

	List with different data types: [1, 'Hello', 3.14, True, [1, 2, 3]]


	Rishabh Rathore
	0827CI211155
	
\end{verbatim}
\newpage


\qs{}{
Write a program to take user input in a list from eval\(\).
}
\sol{}
\begin{lstlisting}[language=Python]

	# Take user input as a string
	user_input_str = input("Enter elements of the list separated by commas: ")

	# Convert the user input string to a list using eval()
	try:
		user_list = eval(user_input_str)
		if not isinstance(user_list, list):
			raise ValueError("Input is not a valid list.")
	except Exception as e:
		print("Error:", e)
	else:
		print("User input list:", user_list)
	print("\n")
	print("Rishabh Rathore")
	print("0827CI211155")
\end{lstlisting}

\begin{verbatim}
	Output:

	Enter elements of the list separated by commas: 12,18,24,7,10
	Error: Input is not a valid list.
	
	
	Rishabh Rathore
	0827CI211155
\end{verbatim}
\newpage


\qs{}{
Write a program to append five elements in a list by using for loop and append\(\).
}
\sol{}
\begin{lstlisting}[language=Python]

	# Initialize an empty list
	my_list = []

	# Append five elements to the list using a for loop and append() method
	for i in range(1, 6):
		my_list.append(i)

	# Print the list
	print("List after appending five elements:", my_list)
	print("\n")
	print("Rishabh Rathore")
	print("0827CI211155")

\end{lstlisting}

\begin{verbatim}
	Output:

	List after appending five elements: [1, 2, 3, 4, 5]


	Rishabh Rathore
	0827CI211155
\end{verbatim}
\newpage


\qs{}{
Write a program to sort a list in both ascending and descending order.
}
\sol{}
\begin{lstlisting}[language=Python]

	# Original list
	my_list = [5, 2, 8, 1, 9]

	# Sort the list in ascending order using sort() method
	my_list.sort()
	print("List sorted in ascending order:", my_list)

	# Sort the list in descending order using sorted() function
	descending_list = sorted(my_list, reverse=True)
	print("List sorted in descending order:", descending_list)
	print("\n")
	print("Rishabh Rathore")
	print("0827CI211155")
\end{lstlisting}

\begin{verbatim}
	Output:

	List sorted in ascending order: [1, 2, 5, 8, 9]
	List sorted in descending order: [9, 8, 5, 2, 1]
	
	
	Rishabh Rathore
	0827CI211155
\end{verbatim}
\newpage


\qs{}{
Write a program to count the occurrences of an element in a list.
}
\sol{}
\begin{lstlisting}[language=Python]

	# Original list
	my_list = [1, 2, 3, 4, 2, 3, 2, 5, 2]

	# Element to count occurrences
	element_to_count = 2

	# Count occurrences of the element in the list
	occurrences = my_list.count(element_to_count)

	# Print the result
	print(f"The element {element_to_count} occurs {occurrences} times in the list.")
	print("\n")
	print("Rishabh Rathore")
	print("0827CI211155")
\end{lstlisting}

\begin{verbatim}
	The element 2 occurs 4 times in the list.


	Rishabh Rathore
	0827CI211155
\end{verbatim}
\newpage


\qs{}{
Write a program to find the index of an element in the list.
}
\sol{}
\begin{lstlisting}[language=Python]
	# Original list
	my_list = [10, 20, 30, 40, 50]

	# Element to find index
	element_to_find = 30

	try:
		# Find the index of the element in the list
		index = my_list.index(element_to_find)
		print(f"The index of {element_to_find} in the list is:", index)
	except ValueError:
		print(f"The element {element_to_find} is not present in the list.")
	print("\n")
	print("Rishabh Rathore")
	print("0827CI211155")
\end{lstlisting}

\begin{verbatim}

	Output:

	The index of 30 in the list is: 2


	Rishabh Rathore
	0827CI211155
\end{verbatim}
\newpage


\qs{}{
Write a program to swap the first and last element of a list.
}
\sol{}
\begin{lstlisting}[language=Python]
	# Original list
	my_list = [10, 20, 30, 40, 50]
	print("List before swaping : ",my_list)
	# Swap the first and last elements of the list
	if len(my_list) >= 2:
		my_list[0], my_list[-1] = my_list[-1], my_list[0]

	# Print the modified list
	print("List after swapping first and last elements:", my_list)
	print("\n")
	print("Rishabh Rathore")
	print("0827CI211155")
\end{lstlisting}

\begin{verbatim}
	Output:

	List before swaping :  [10, 20, 30, 40, 50]
	List after swapping first and last elements: [50, 20, 30, 40, 10]
	
	
	Rishabh Rathore
	0827CI211155
\end{verbatim}
\newpage


\qs{}{
Write a program to swap two no in list with given position.
}
\sol{}
\begin{lstlisting}[language=Python]

	def swap_elements(lst, pos1, pos2):
		"""
		Function to swap two elements in a list at given positions.

		Parameters:
		lst (list): The input list.
		pos1 (int): The position of the first element to swap.
		pos2 (int): The position of the second element to swap.

		Returns:
		list: The list with elements swapped.
		"""
		# Check if positions are within the range of the list
		if 0 <= pos1 < len(lst) and 0 <= pos2 < len(lst):
			# Swap the elements
			lst[pos1], lst[pos2] = lst[pos2], lst[pos1]
		else:
			print("Positions are out of range. Swapping not possible.")
		return lst

	# Original list
	my_list = [10, 20, 30, 40, 50]

	# Positions to swap
	position1 = 1
	position2 = 3

	# Swap elements at given positions
	modified_list = swap_elements(my_list, position1, position2)

	# Print the modified list
	print("List after swapping elements at positions", position1, "and", position2, ":", modified_list)
	print("\n")
	print("Rishabh Rathore")
	print("0827CI211155")
\end{lstlisting}

\begin{verbatim}
	
	Output:

	List after swapping elements at positions 1 and 3 : [10, 40, 30, 20, 50]


	Rishabh Rathore
	0827CI211155
\end{verbatim}
\newpage


\qs{}{
Write a program to check whether a no is in the list or not sublist even and odd.
}
\sol{}
\begin{lstlisting}[language=Python]

	# Original list
	my_list = [1, 2, 3, 4, 5, 6, 7, 8, 9]

	# Number to check
	number_to_check = 5

	# Check if the number is present in the list
	if number_to_check in my_list:
		print(f"The number {number_to_check} is present in the list.")
	else:
		print(f"The number {number_to_check} is not present in the list.")

	# Separate the list into two sublists containing even and odd numbers
	even_numbers = []
	odd_numbers = []

	for num in my_list:
		if num % 2 == 0:
			even_numbers.append(num)
		else:
			odd_numbers.append(num)

	# Print the sublists
	print("Even numbers in the list:", even_numbers)
	print("Odd numbers in the list:", odd_numbers)
	print("\n")
	print("Rishabh Rathore")
	print("0827CI211155")
\end{lstlisting}

\begin{verbatim}
	Output:

	The number 5 is present in the list.
	Even numbers in the list: [2, 4, 6, 8]
	Odd numbers in the list: [1, 3, 5, 7, 9]
	
	
	Rishabh Rathore
	0827CI211155
\end{verbatim}
\newpage


\qs{}{
Write a program to demonstrate the difference between remove and pop method in a list.
}
\sol{}
\begin{lstlisting}[language=Python]

	# Original list
	my_list = [1, 2, 3, 4, 5]

	# Demonstrate remove() method
	removed_element = 3
	print("Original list before using remove():", my_list)
	my_list.remove(removed_element)
	print("List after using remove():", my_list)
	print("Removed element:", removed_element)

	# Demonstrate pop() method
	index_to_pop = 2
	popped_element = my_list.pop(index_to_pop)
	print("Original list before using pop():", my_list)
	print("Popped element at index", index_to_pop, ":", popped_element)
	print("\n")
	print("Rishabh Rathore")
	print("0827CI211155")

\end{lstlisting}

\begin{verbatim}
	Output:

	Original list before using remove(): [1, 2, 3, 4, 5]
	List after using remove(): [1, 2, 4, 5]
	Removed element: 3
	Original list before using pop(): [1, 2, 5]
	Popped element at index 2 : 4
	
	
	Rishabh Rathore
	0827CI211155
\end{verbatim}
\newpage


\qs{}{
Write a program to insert an element into a list at a position given by the user.
}
\sol{}
\begin{lstlisting}[language=Python]
	# Original list
	my_list = [1, 2, 3, 4, 5]

	# Get element and position from the user
	element_to_insert = int(input("Enter the element to insert: "))
	position = int(input("Enter the position to insert the element: "))

	# Check if the position is within the range of the list
	if 0 <= position <= len(my_list):
		# Insert the element at the specified position
		my_list.insert(position, element_to_insert)
		print("List after inserting the element:", my_list)
	else:
		print("Invalid position. Element cannot be inserted.")
	print("\n")
	print("Rishabh Rathore")
	print("0827CI211155")
\end{lstlisting}

\begin{verbatim}
	Output:

	Enter the element to insert: 12
	Enter the position to insert the element: 2
	List after inserting the element: [1, 2, 12, 3, 4, 5]
	
	
	Rishabh Rathore
	0827CI211155
\end{verbatim}
\newpage

% q 1
\qs{}{Explain the difference between the list, tuple , set and dictionary ?}
\sol 
\begin{lstlisting}[language=Python]
	print("Rishabh Rathore")
	print("0827CI211155")

	# List
	my_list = [1, 2, 3, 4, 5]

	print(my_list)
	# Lists maintain order
	for item in my_list:
    	print(item)

	# Tuple
	my_tuple = (1, 2, 3, 4, 5)

	# Tuples are immutable - attempting to modify will raise an error

	for item in my_tuple:
    	print(item)

	# Set
	my_set = {1, 2, 3, 4, 5}

	# Sets are mutable
	my_set.add(6)
	print(my_set)  # Output: {1, 2, 3, 4, 5, 6}

	# Sets do not maintain order
	for item in my_set:
 	   print(item)

	# Dictionary
	my_dict = {'name': 'John', 'age': 30, 'city': 'New York'}

	# Dictionaries are mutable
	my_dict['age'] = 35
	print(my_dict)  # Output: {'name': 'John', 'age': 35, 'city': 'New York'}

	# Dictionaries do not maintain order (prior to Python 3.7)
	for key, value in my_dict.items():
    	print(key, ':', value)


\end{lstlisting}

\begin{verbatim}
	Output:

	Rishabh Rathore
	0827CI211155


	[10, 2, 3, 4, 5]

	10
	2
	3
	4
	5

	1
	2
	3
	4
	5

	{1,2,3,4,5,6}

	1
	2
	3
	4
	5
	6

	{'name': 'John', 'age': 35, 'city': 'New York'}

	name : John
	age : 35
	city : New York

\end{verbatim}



\newpage

% 2
\qs{}{Write a program to Counts the number of occurrences of item 50 from a tuple.}
\sol 
\begin{lstlisting}[language=Python]
	print("Rishabh Rathore")
	print("0827CI211155")

	def count_occurrences(tuple, item):
    count = 0
    for element in tuple:
        if element == item:
            count += 1
    return count

	# Example tuple
	example_tuple = (10, 20, 30, 40, 50, 50, 50, 60, 70)

	# Count occurrences of item 50
	occurrences = count_occurrences(example_tuple, 50)
	print("Number of occurrences of item 50:", occurrences)


\end{lstlisting}

\begin{verbatim}
	Output:

	Rishabh Rathore
	0827CI211155
	Number of occurrences of item 50: 3
\end{verbatim}



\newpage

% 3
\qs{}{ Write a program to Check if all items in the tuple are the same}
\sol 
\begin{lstlisting}[language=Python]
	print("Rishabh Rathore")
	print("0827CI211155")
	def all_same(tuple):
    # Check if all elements are the same
    return all(element == tuple[0] for element in tuple)

	# Example tuples
	tuple1 = (10, 10, 10, 10)
	tuple2 = (10, 20, 10, 10)

	# Check if all items in tuple1 are the same
	if all_same(tuple1):
		print("All items in tuple1 are the same")
	else:
		print("Not all items in tuple1 are the same")

	# Check if all items in tuple2 are the same
	if all_same(tuple2):
		print("All items in tuple2 are the same")
	else:
		print("Not all items in tuple2 are the same")
  

\end{lstlisting}

\begin{verbatim}
	Output:

	Rishabh Rathore
	0827CI211155
	All items in tuple1 are the same
	Not all items in tuple2 are the same

\end{verbatim}




\newpage

% q 4
\qs{}{Write a Program to create a tuple which is having integer,float,list and tuple as its elements and print an element present in list of these tuple}
\sol 
\begin{lstlisting}[language=Python]
	print("Rishabh Rathore")
	print("0827CI211155")

	# Define the tuple with different types of elements
	mixed_tuple = (10, 3.14, [1, 2, 3], ('a', 'b', 'c'))

	# Function to print an element from a list within the tuple
	def print_list_element(tuple):
		# Get the list from the tuple
		list_element = tuple[2]
		# Print an element from the list
		print("Element from the list within the tuple:", list_element[1])

	# Call the function to print an element from the list within the tuple
	print_list_element(mixed_tuple)

\end{lstlisting}

\begin{verbatim}
	Output:

	Rishabh Rathore
	0827CI211155
	Element from the list within the tuple: 2

\end{verbatim}



\newpage

% q 5
\qs{}{Write a program to create a tuple with the name of 10 cities of India Check whether a City is present in the tuple or not}
\sol 
\begin{lstlisting}[language=Python]
	print("Rishabh Rathore")
	print("0827CI211155")

	def city_present(city, cities_tuple):
    # Check if the city is present in the tuple
    return city in cities_tuple

	# Create a tuple with the names of 10 cities of India
	indian_cities = ('Delhi', 'Mumbai', 'Bangalore', 
					'Kolkata','Chennai', 'Hyderabad',
					 'Pune', 'Ahmedabad', 'Jaipur', 
					 'Lucknow')

	# Input city to check
	city_to_check = input("Enter the name of the city to check: ")

	# Check if the city is present in the tuple
	if city_present(city_to_check, indian_cities):
		print(city_to_check, "is present in the tuple of Indian cities.")
	else:
		print(city_to_check, "is not present in the tuple of Indian cities.")
\end{lstlisting}

\begin{verbatim}
	Output:

	Rishabh Rathore
	0827CI211155
	Enter the name of the city to check: Kolkata
	Kolkata is present in the tuple of Indian cities.

\end{verbatim}


\newpage

% q 6
\qs{}{Write a program to get the number of occurrence of a word in tuple.}
\sol 
\begin{lstlisting}[language=Python]
	print("Rishabh Rathore")
	print("0827CI211155")

	def count_word_occurrences(word, tuple_of_strings):
    count = 0
    for string in tuple_of_strings:
        # Split the string into words and count occurrences of the target word
        count += string.count(word)
    return count

	# Example tuple of strings
	tuple_of_strings = ("apple banana", "banana orange", 
					"banana apple", "orange mango banana")

	# Word to count occurrences of
	word_to_count = input("Enter the word to count occurrences: ")

	# Count occurrences of the word in the tuple
	occurrences = count_word_occurrences(word_to_count, tuple_of_strings)
	print("Number of occurrences of", 
		word_to_count, "in the tuple:",
		 occurrences)

\end{lstlisting}

\begin{verbatim}
	Output:

	Rishabh Rathore
	0827CI211155
	Enter the word to count occurrences: apple
	Number of occurrences of apple in the tuple: 2



\end{verbatim}


\newpage

% q 7
\qs{}{Write a program to get the index of a word in tuple}
\sol 
\begin{lstlisting}[language=Python]
	print("Rishabh Rathore")
	print("0827CI211155")

	def get_word_index(word, tuple_of_strings):
    indexes = []
    for i, string in enumerate(tuple_of_strings):
        # Split the string into words and check if the word is present
        if word in string:
            indexes.append(i)
    return indexes

	# Example tuple of strings
	tuple_of_strings = ("apple banana", "banana orange",
				 "banana apple", "orange mango banana")

	# Word to get the index of
	word_to_find = input("Enter the word to find its index: ")

	# Get the index of the word in the tuple
	word_indexes = get_word_index(word_to_find, tuple_of_strings)

	if word_indexes:
		print("Indexes of", word_to_find, "in the tuple:", word_indexes)
	else:
		print(word_to_find, "not found in the tuple.")

\end{lstlisting}

\begin{verbatim}
	Output:

	Rishabh Rathore
	0827CI211155
	Enter the word to find its index: the
	the not found in the tuple.



\end{verbatim}


\newpage

% q 8
\qs{}{Write a program to create four tuples viz roll no. name. CGPA and SGPA of students. Print individual student's details using these 4 tuples.}
\sol 
\begin{lstlisting}[language=Python]
	print("Rishabh Rathore")
	print("0827CI211155")

	
	# Function to print individual student's details
	def print_student_details(roll_no, name, cgpa, sgpa):
		print("Roll No:", roll_no)
		print("Name:", name)
		print("CGPA:", cgpa)
		print("SGPA:", sgpa)
		print()

	# Create tuples for student details
	roll_numbers = (101, 102, 103, 104)
	names = ("John Doe", "Jane Smith", "Alice Johnson", "Bob Brown")
	cgpa = (3.8, 3.9, 3.7, 4.0)
	sgpa = (4.0, 3.9, 3.8, 4.0)

	# Print individual student's details
	for i in range(len(roll_numbers)):
		print("Student", i+1, "details:")
		print_student_details(roll_numbers[i], names[i], cgpa[i], sgpa[i])
	print("Ritesh Telkar")
	print("0827CI211158")
\end{lstlisting}

\begin{verbatim}
	Output:

	Rishabh Rathore
	0827CI211155
	Student 1 details:
	Roll No: 101
	Name: John Doe
	CGPA: 3.8
	SGPA: 4.0

	Student 2 details:
	Roll No: 102
	Name: Jane Smith
	CGPA: 3.9
	SGPA: 3.9

	Student 3 details:
	Roll No: 103
	Name: Alice Johnson
	CGPA: 3.7
	SGPA: 3.8

	Student 4 details:
	Roll No: 104
	Name: Bob Brown
	CGPA: 4.0
	SGPA: 4.0


\end{verbatim}



\newpage

% q 9
\qs{}{write a Python program to create a set.}
\sol 
\begin{lstlisting}[language=Python]
	print("Rishabh Rathore")
	print("0827CI211155")

	# Create a set
	my_set = {1, 2, 3, 4, 5}

	# Print the set
	print("Set:", my_set)
\end{lstlisting}

\begin{verbatim}
	Output:

	Rishabh Rathore
	0827CI211155
	Set: {1, 2, 3, 4, 5}

\end{verbatim}


\newpage


% q 10
\qs{}{Write a Python program to iterate over sets.}
\sol 
\begin{lstlisting}[language=Python]
	print("Rishabh Rathore")
	print("0827CI211155")

	# Define a set
	my_set = {1, 2, 3, 4, 5}

	# Iterate over the set and print each element
	print("Elements of the set:")
	for element in my_set:
		print(element)
\end{lstlisting}

\begin{verbatim}
	Output:

	Rishabh Rathore
	0827CI211155
	Elements of the set:
	1
	2
	3
	4
	5
	
	
\end{verbatim}


\newpage


% q 11
\qs{}{Write a Python program to add member(s) to a set.}
\sol 
\begin{lstlisting}[language=Python]
	print("Rishabh Rathore")
	print("0827CI211155")

	# Define a set
	my_set = {1, 2, 3, 4, 5}
	
	# Print the initial set
	print("Initial set:", my_set)
	
	# Add a single member to the set
	my_set.add(6)
	print("Set after adding a single member:", my_set)
	
	# Add multiple members to the set using update() method
	my_set.update([7, 8, 9])
	print("Set after adding multiple members:", my_set)
\end{lstlisting}

\begin{verbatim}
	Output:

	Rishabh Rathore
	0827CI211155
	Initial set: {1, 2, 3, 4, 5}
	Set after adding a single member: {1, 2, 3, 4, 5, 6}
	Set after adding multiple members: {1, 2, 3, 4, 5, 6, 7, 8, 9}

	

\end{verbatim}


\newpage


% q 12
\qs{}{Write a Python program to remove Item(s) from a given set.}
\sol 
\begin{lstlisting}[language=Python]
	print("Rishabh Rathore")
	print("0827CI211155")

	my_set = {1, 2, 3, 4, 5}

	# Print the initial set
	print("Initial set:", my_set)
	
	# Remove a single item from the set using discard() method
	my_set.discard(3)
	print("Set after removing a single item:", my_set)
	
	# Remove multiple items from the set using discard() method
	items_to_remove = {1, 4}
	my_set.difference_update(items_to_remove)
	print("Set after removing multiple items:", my_set)
\end{lstlisting}

\begin{verbatim}
	Output:

	Rishabh Rathore
	0827CI211155
	Initial set: {1, 2, 3, 4, 5}
	Set after removing a single item: {1, 2, 4, 5}
	Set after removing multiple items: {2, 5}

\end{verbatim}


\newpage


% q 13
\qs{}{Write a Python program to remove an item from a set if it is}
\sol 
\begin{lstlisting}[language=Python]
	print("Rishabh Rathore")
	print("0827CI211155")

	def remove_item(set_name, item):
    if item in set_name:
        set_name.remove(item)
        print(f"Item '{item}' removed from the set")
    else:
        print(f"Item '{item}' is not present in the set")

	# Define a set
	my_set = {1, 2, 3, 4, 5}

	# Print the initial set
	print("Initial set:", my_set)

	# Remove an item from the set
	item_to_remove = int(input("Enter the item to remove: "))
	remove_item(my_set, item_to_remove)

	# Print the set after removal
	print("Set after removal:", my_set)
\end{lstlisting}

\begin{verbatim}
	Output:

	Rishabh Rathore
	0827CI211155
	Initial set: {1, 2, 3, 4, 5}
	Enter the item to remove: 3
	Item '3' removed from the set
	Set after removal: {1, 2, 4, 5}
	
	

\end{verbatim}


\newpage


% q 13
\qs{}{Write a Python program to create an intersection of sets.}
\sol 
\begin{lstlisting}[language=Python]
	print("Rishabh Rathore")
	print("0827CI211155")

	# Define two sets
	set1 = {1, 2, 3, 4, 5}
	set2 = {4, 5, 6, 7, 8}
	
	# Find the intersection of the sets using the intersection() method
	intersection_set = set1.intersection(set2)
	
	# Print the intersection set
	print("Intersection of set1 and set2:", intersection_set)
\end{lstlisting}

\begin{verbatim}
	Output:

	Rishabh Rathore
	0827CI211155
	Intersection of set1 and set2: {4, 5}
	
\end{verbatim}


\newpage


\qs{}{Write a Python program to create a union of sets.}
\sol 
\begin{lstlisting}[language=Python]
	print("Rishabh Rathore")
	print("0827CI211155")

	# Define two sets
	set1 = {1, 2, 3, 4, 5}
	set2 = {4, 5, 6, 7, 8}
	
	# Find the union of the sets using the union() method
	union_set = set1.union(set2)
	
	# Print the union set
	print("Union of set1 and set2:", union_set)
\end{lstlisting}

\begin{verbatim}
	Output:

	Rishabh Rathore
	0827CI211155
	Union of set1 and set2: {1, 2, 3, 4, 5, 6, 7, 8}

\end{verbatim}


\newpage


\qs{}{Write a Python program to create set difference}
\sol 
\begin{lstlisting}[language=Python]
	print("Rishabh Rathore")
	print("0827CI211155")

	# Define two sets
	set1 = {1, 2, 3, 4, 5}
	set2 = {4, 5, 6, 7, 8}
	
	# Find the set difference using the difference() method
	difference_set = set1.difference(set2)
	
	# Print the set difference
	print("Set difference (set1 - set2):", difference_set)
\end{lstlisting}

\begin{verbatim}
	Output:

	Rishabh Rathore
	0827CI211155
	Set difference (set1 - set2): {1, 2, 3}

\end{verbatim}


\newpage


\qs{}{Write a Python program to create a symmetric difference.}
\sol 
\begin{lstlisting}[language=Python]
	print("Rishabh Rathore")
	print("0827CI211155")

	# Define two sets
	set1 = {1, 2, 3, 4, 5}
	set2 = {4, 5, 6, 7, 8}
	
	# Find the symmetric difference using the symmetric_difference() method
	symmetric_difference_set = set1.symmetric_difference(set2)
	
	# Print the symmetric difference
	print("Symmetric difference of set1 and set2:", symmetric_difference_set)
\end{lstlisting}

\begin{verbatim}
	Output:

	Rishabh Rathore
	0827CI211155
	Symmetric difference of set1 and set2: {1, 2, 3, 6, 7, 8}
\end{verbatim}


\newpage


\qs{}{write a Python program to check if a set is a subset of another set.}
\sol 
\begin{lstlisting}[language=Python]
	print("Rishabh Rathore")
	print("0827CI211155")

	def is_subset(set1, set2):
    # Check if set1 is a subset of set2
    return set1.issubset(set2)

	# Example usage:
	set1 = {1, 2, 3}
	set2 = {1, 2, 3, 4, 5}

	if is_subset(set1, set2):
		print("set1 is a subset of set2")
	else:
		print("set1 is not a subset of set2")
\end{lstlisting}

\begin{verbatim}
	Output:

	Rishabh Rathore
	0827CI211155
	set1 is a subset of set2

\end{verbatim}


\newpage


\qs{}{Write a Python program to remove all elements from a given set}
\sol 
\begin{lstlisting}[language=Python]
	print("Rishabh Rathore")
	print("0827CI211155")

	def remove_all_elements(some_set):
    some_set.clear()  # Clears all elements from the set

	# Example usage:
	my_set = {1, 2, 3, 4, 5}
	print("Before removing elements:", my_set)

	remove_all_elements(my_set)
	print("After removing elements:", my_set)
\end{lstlisting}

\begin{verbatim}
	Output:

	Rishabh Rathore
	0827CI211155
	Before removing elements: {1, 2, 3, 4, 5}
	After removing elements: set()

\end{verbatim}


\newpage


\qs{}{Write a Python program to find the maximum and minimum values in a set}
\sol 
\begin{lstlisting}[language=Python]
	print("Rishabh Rathore")
	print("0827CI211155")

	def find_max_min(some_set):
    if len(some_set) == 0:
        print("The set is empty.")
    else:
        max_value = max(some_set)
        min_value = min(some_set)
        return max_value, min_value

	# Example usage:
	my_set = {5, 3, 9, 2, 8, 1}

	result = find_max_min(my_set)
	if result:
		max_val, min_val = result
		print("Maximum value:", max_val)
		print("Minimum value:", min_val)
\end{lstlisting}

\begin{verbatim}
	Output:

	Rishabh Rathore
	0827CI211155
	Maximum value: 9
	Minimum value: 1

\end{verbatim}


\newpage


\qs{}{write a Python program to find length of a set}
\sol 
\begin{lstlisting}[language=Python]
	print("Rishabh Rathore")
	print("0827CI211155")

	def find_set_length(some_set):
    return len(some_set)

	# Example usage:
	my_set = {1, 2, 3, 4, 5}
	set_length = find_set_length(my_set)
	print("Length of the set:", set_length)
\end{lstlisting}

\begin{verbatim}
	Output:

	Rishabh Rathore
	0827CI211155
	Length of the set: 5

\end{verbatim}


\newpage


\qs{}{Write a Python program to check if a given value is present in a set or not.}
\sol 
\begin{lstlisting}[language=Python]
	print("Rishabh Rathore")
	print("0827CI211155")

	def check_value_in_set(some_set, value):
    return value in some_set

	# Example usage:
	my_set = {1, 2, 3, 4, 5}
	value_to_check = 3

	if check_value_in_set(my_set, value_to_check):
		print("The value", value_to_check, "is present in the set.")
	else:
		print("The value", value_to_check, "is not present in the set.")
\end{lstlisting}

\begin{verbatim}
	Output:

	Rishabh Rathore
	0827CI211155
	The value 3 is present in the set

\end{verbatim}


\newpage


\qs{}{Write a Python program to check if two given sets have no elements in common.}
\sol 
\begin{lstlisting}[language=Python]
	print("Rishabh Rathore")
	print("0827CI211155")

	def no_common_elements(set1, set2):
    return set1.isdisjoint(set2)

	# Example usage:
	set1 = {1, 2, 3}
	set2 = {4, 5, 6}

	if no_common_elements(set1, set2):
		print("The sets have no common elements.")
	else:
		print("The sets have common elements.")

\end{lstlisting}

\begin{verbatim}
	Output:

	Rishabh Rathore
	0827CI211155
	The sets have no common elements.


\end{verbatim}


\newpage


\qs{}{ Write a Python program to create a Dictionary of student information. take roll no as key}
\sol 
\begin{lstlisting}[language=Python]
	print("Rishabh Rathore")
	print("0827CI211155")
	def create_student_dictionary():
    num_students = int(input("Enter the number of students: "))
    student_dict = {}

    for _ in range(num_students):
        roll_no = input("Enter roll number: ")
        name = input("Enter name: ")
        age = int(input("Enter age: "))
        grade = input("Enter grade: ")
        student_dict[roll_no] = {"Name": name, "Age": age, "Grade": grade}

    return student_dict

	def main():
		student_info = create_student_dictionary()
		print("\nStudent Information:")
		for roll_no, info in student_info.items():
			print("for student ")
			print(f"Roll No: {roll_no}")
			print(f"Name: {info['Name']}")
			print(f"Age: {info['Age']}")
			print(f"Grade: {info['Grade']}")
			print()

	if __name__ == "__main__":
		main()
  

\end{lstlisting}

\begin{verbatim}
	Output:

	Rishabh Rathore
	0827CI211155
	Enter the number of students: 3
	Enter roll number: 101
	Enter name: abhay
	Enter age: 23
	Enter grade: A
	Enter roll number: 102
	Enter name: akaay
	Enter age: 22
	Enter grade: C
	Enter roll number: 103
	Enter name: asif
	Enter age: 21
	Enter grade: B
	
	Student Information:
	Roll No: 101
	Name: abhay
	Age: 23
	Grade: A
	
	Roll No: 102
	Name: akaay
	Age: 22
	Grade: C
	
	Roll No: 103
	Name: asif
	Age: 21
	Grade: B
	

\end{verbatim}


\newpage


\qs{}{Write a Python progarm to delete a key from the dictionary }
\sol 
\begin{lstlisting}[language=Python]
	print("Rishabh Rathore")
	print("0827CI211155")

	def delete_key_from_dict(student_dict, roll_no):
    if roll_no in student_dict:
        del student_dict[roll_no]
        print(f"Key '{roll_no}' deleted successfully.")
    else:
        print(f"Key '{roll_no}' not found in the dictionary.")

	def main():
		# Example student dictionary
		student_dict = {
			"001": {"Name": "Alice", "Age": 18, "Grade": "A"},
			"002": {"Name": "Bob", "Age": 19, "Grade": "B"},
			"003": {"Name": "Charlie", "Age": 20, "Grade": "C"}
		}

		print("Original Dictionary:")
		print(student_dict)

		roll_no_to_delete = input("\nEnter the roll number to delete: ")
		delete_key_from_dict(student_dict, roll_no_to_delete)

		print("\nUpdated Dictionary:")
		print(student_dict)

	if __name__ == "__main__":
		main()

\end{lstlisting}

\begin{verbatim}
	Output:

	Rishabh Rathore
	0827CI211155
	Original Dictionary:
	{'001': {'Name': 'Alice', 'Age': 18, 'Grade': 'A'}, 
	 '002': {'Name': 'Bob', 'Age': 19, 'Grade': 'B'},
	 '003': {'Name': 'Charlie', 'Age': 20, 'Grade': 'C'}}

	Enter the roll number to delete: 003
	Key '003' deleted successfully.

	Updated Dictionary:
	{'001': {'Name': 'Alice', 'Age': 18, 'Grade': 'A'}, 
	'002': {'Name': 'Bob', 'Age': 19, 'Grade': 'B'}}
	

\end{verbatim}


\newpage


\qs{}{Write a Python program to add or update the data.}
\sol 
\begin{lstlisting}[language=Python]
	print("Rishabh Rathore")
	print("0827CI211155")

	def add_or_update_data(dictionary, key, value):
    """
    Function to add or update data in a dictionary.

    Parameters:
    dictionary (dict): The dictionary to modify.
    key: The key to add/update in the dictionary.
    value: The value corresponding to the key.

    Returns:
    None
    """
    dictionary[key] = value


	my_dictionary = {'a': 1, 'b': 2, 'c': 3}

	print("Original Dictionary:", my_dictionary)


	add_or_update_data(my_dictionary, 'd', 4)
	print("Dictionary after adding 'd':", my_dictionary)


	add_or_update_data(my_dictionary, 'b', 5)
	print("Dictionary after updating 'b':", my_dictionary)

  

\end{lstlisting}

\begin{verbatim}
	Output:

	Rishabh Rathore
	0827CI211155
	Original Dictionary: {'a': 1, 'b': 2, 'c': 3}
	Dictionary after adding 'd': {'a': 1, 'b': 2, 'c': 3, 'd': 4}
	Dictionary after updating 'b': {'a': 1, 'b': 5, 'c': 3, 'd': 4}

	

\end{verbatim}


\newpage


\qs{}{Write a Python script to concatenate the following dictionaries to create a new one sample Dictionary:dic1={1:10,2:20}, dic2={3:30,4:40}}
\sol 
\begin{lstlisting}[language=Python]
	print("Rishabh Rathore")
	print("0827CI211155")
	def concatenate_dicts(*dicts):
    """
    Concatenate multiple dictionaries into a new one.

    Parameters:
    *dicts: Variable number of dictionaries to concatenate.

    Returns:
    dict: The concatenated dictionary.
    """
    concatenated_dict = {}
    for d in dicts:
        concatenated_dict.update(d)
    return concatenated_dict

	# Sample dictionaries
	dic1 = {1: 10, 2: 20}
	dic2 = {3: 30, 4: 40}

	# Concatenate dictionaries
	concatenated_dict = concatenate_dicts(dic1, dic2)

	# Output the concatenated dictionary
	print("Concatenated Dictionary:", concatenated_dict)


\end{lstlisting}

\begin{verbatim}
	Output:

	Rishabh Rathore
	0827CI211155
	Concatenated Dictionary: {1: 10, 2: 20, 3: 30, 4: 40}

	

\end{verbatim}


\newpage


\qs{}{Write a Python script to check whether a given key already exists in a dictionary}
\sol 
\begin{lstlisting}[language=Python]
	print("Rishabh Rathore")
	print("0827CI211155")
	def check_key_existence(dictionary, key):
    """
    Check whether a given key exists in a dictionary.

    Parameters:
    dictionary (dict): The dictionary to check.
    key: The key to check for existence.

    Returns:
    bool: True if the key exists, False otherwise.
    """
    return key in dictionary

	# Sample dictionary
	my_dict = {'a': 1, 'b': 2, 'c': 3}

	# Key to check
	key_to_check = 'b'

	# Check if the key exists in the dictionary
	if check_key_existence(my_dict, key_to_check):
		print(f"The key '{key_to_check}' exists in the dictionary.")
	else:
		print(f"The key '{key_to_check}' does not exist in the dictionary.")
\end{lstlisting}

\begin{verbatim}
	Output:

	Rishabh Rathore
	0827CI211155
	The key 'b' exists in the dictionary.	

\end{verbatim}



\newpage


\qs{}{Write a Python program to iterate over dictionaries using for loops}
\sol 
\begin{lstlisting}[language=Python]
	print("Rishabh Rathore")
	print("0827CI211155")

	# Sample dictionary
	my_dict = {'a': 1, 'b': 2, 'c': 3}
	
	# Iterate over keys
	print("Iterating over keys:")
	for key in my_dict:
		print(key)
	
	# Iterate over values
	print("\nIterating over values:")
	for value in my_dict.values():
		print(value)
	
	# Iterate over key-value pairs
	print("\nIterating over key-value pairs:")
	for key, value in my_dict.items():
		print(key, "->", value)
	

\end{lstlisting}

\begin{verbatim}
	Output:

	Rishabh Rathore
	0827CI211155
	Iterating over keys:
	a
	b
	c
	
	Iterating over values:
	1
	2
	3
	
	Iterating over key-value pairs:
	a -> 1
	b -> 2
	c -> 3
	
	
	
	

\end{verbatim}


\newpage


\qs{}{Write a Python script to generate and print a dictionary that contains a number (between I and n) in the form (x, x*x).Sample dictionary (n=5): Expected output: {1:1,2:4,3:9,4:16,5:25}}
\sol 
\begin{lstlisting}[language=Python]
	print("Rishabh Rathore")
	print("0827CI211155")

	def generate_squared_dict(n):
    """
    Generate a dictionary containing numbers and their squares from 1 to n.

    Parameters:
    n (int): The maximum number to include in the dictionary.

    Returns:
    dict: The generated dictionary.
    """
    squared_dict = {}
    for x in range(1, n+1):
        squared_dict[x] = x * x
    return squared_dict

	# Sample value of n
	n = 8

	# Generate the dictionary
	result_dict = generate_squared_dict(n)

	# Print the generated dictionary
	print("Generated Dictionary:", result_dict)

\end{lstlisting}

\begin{verbatim}
	Output:

	Rishabh Rathore
	0827CI211155
	Generated Dictionary: {1: 1, 2: 4, 3: 9, 4: 16, 5: 25, 6: 36, 7: 49, 8: 64}


\end{verbatim}


\newpage


\qs{}{Write a python function to calculate Sum of two variables.}
\sol 
\begin{lstlisting}[language=Python]
	print("Rishabh Rathore")
	print("0827CI211155")
	def sum_of_two_variables(a, b):
    """
    Calculate the sum of two variables.

    Parameters:
    a: The first variable.
    b: The second variable.

    Returns:
    The sum of a and b.
    """
    return a + b

	# Example usage:
	result = sum_of_two_variables(5, 7)
	print("Sum of the two variables:", result)
  

\end{lstlisting}

\begin{verbatim}
	Output:

	Rishabh Rathore
	0827CI211155
	Sum of the two variables: 12


\end{verbatim}


\newpage


\qs{}{Write a Python function to show the use of Default Parameter}
\sol 
\begin{lstlisting}[language=Python]
	print("Rishabh Rathore")
	print("0827CI211155")
	def greet(name, greeting="Hello"):
    """
    Function to greet a person with a specified greeting.

    Parameters:
    name: The name of the person to greet.
    greeting (optional): 
	The greeting to use. Defaults to "Hello" if not specified.

    Returns:
    str: The greeting message.
    """
    return f"{greeting}, {name}!"

	# Example usage:
	print(greet("Ram"))  # Uses the default greeting "Hello"

\end{lstlisting}

\begin{verbatim}
	Output:

	Rishabh Rathore
	0827CI211155
	Hello, Ram!
	Good morning, Siya!
	
	
\end{verbatim}


\newpage


\qs{}{Write a function to find the maximum of three numbers}
\sol 
\begin{lstlisting}[language=Python]
	print("Rishabh Rathore")
	print("0827CI211155")
	def find_maximum(a, b, c):
    """
    Function to find the maximum of three numbers.

    Parameters:
    a: The first number.
    b: The second number.
    c: The third number.

    Returns:
    The maximum of the three numbers.
    """
    if a >= b:
        if a >= c:
            return a
        else:
            return c
    else:
        if b >= c:
            return b
        else:
            return c

	# Example usage:
	result = find_maximum(10, 5, 8)
	print("Maximum of the three numbers:", result)
  

\end{lstlisting}

\begin{verbatim}
	Output:

	Rishabh Rathore
	0827CI211155
	Maximum of the three numbers: 10

\end{verbatim}


\newpage


\qs{}{Write a python function to use arbitrary argument}
\sol 
\begin{lstlisting}[language=Python]
	print("Rishabh Rathore")
	print("0827CI211155")
	def print_arguments(*args):
    """
    Function to print arbitrary arguments passed to it.

    Parameters:
    *args: Arbitrary number of arguments.

    Returns:
    None
    """
    for arg in args:
        print(arg)

	# Example usage:
	print_arguments(1, 2, 3)
	print_arguments('Hello', 'world', '!')
  

\end{lstlisting}

\begin{verbatim}
	Output:

	Rishabh Rathore
	0827CI211155
	1
	2
	3
	Hello
	world
	!
		

\end{verbatim}


\newpage


\qs{}{Write a Python function to reverse a string}
\sol 
\begin{lstlisting}[language=Python]
	print("Rishabh Rathore")
	print("0827CI211155")
	def reverse_string(input_string):
    """
    Function to reverse a given string.

    Parameters:
    input_string (str): The string to be reversed.

    Returns:
    str: The reversed string.
    """
    return input_string[::-1]

	# Example usage:
	original_string = "hello"
	reversed_string = reverse_string(original_string)
	print("Original string:", original_string)
	print("Reversed string:", reversed_string)

\end{lstlisting}

\begin{verbatim}
	Output:

	Rishabh Rathore
	0827CI211155
	Original string: hello
	Reversed string: olleh

\end{verbatim}


\newpage


\qs{}{Write a Python function to print the document string}
\sol 
\begin{lstlisting}[language=Python]
	print("Rishabh Rathore")
	print("0827CI211155")
	def print_docstring(func):
    """
    Function to print the docstring of a given function.

    Parameters:
    func: The function whose docstring is to be printed.

    Returns:
    None
    """
    print(func.__doc__)

	# Example usage:
	def example_function():
		"""
		This is an example function.
		It does nothing.
		"""
		pass

	print_docstring(example_function)

\end{lstlisting}

\begin{verbatim}
	Output:

	Rishabh Rathore
	0827CI211155

    This is an example function.
    It does nothing.


\end{verbatim}


\newpage


\qs{}{Write a Python function to demonstrate the scope of local and global variable.}
\sol 
\begin{lstlisting}[language=Python]
	print("Rishabh Rathore")
	print("0827CI211155")
	global_variable = "I am global"

	def demonstrate_scope():
		"""
		Function to demonstrate the scope of local and global variables.
		"""
		local_variable = "I am local"
		print("Inside the function:")
		print("Local variable:", local_variable)
		print("Global variable:", global_variable)
	
	# Call the function
	demonstrate_scope()
	
	# Attempt to access local_variable 
	# outside the function - this will raise an error
	# print("Outside the function:")
	# print("Local variable:", local_variable)
	
	# Access global_variable outside the function
	print("Accessing global variable outside the function:", global_variable)


\end{lstlisting}

\begin{verbatim}
	Output:

	Rishabh Rathore
	0827CI211155
	Inside the function:
	Local variable: I am local
	Global variable: I am global
	Accessing global variable outside the function: I am global

\end{verbatim}


\newpage


\qs{}{Write a Python function to calculate the factorial of a number (a nonnegative integer). The function accepts the number as an argument}
\sol 
\begin{lstlisting}[language=Python]
	print("Rishabh Rathore")
	print("0827CI211155")
	def factorial(n):
    """
    Function to calculate the factorial of a nonnegative integer.

    Parameters:
    n (int): The number whose factorial is to be calculated.

    Returns:
    int: The factorial of the input number.
    """
    if n < 0:
        return "Factorial is not defined for negative numbers"
    elif n == 0 or n == 1:
        return 1
    else:
        result = 1
        for i in range(2, n + 1):
            result *= i
        return result

	# Example usage:
	number = 5
	print(f"The factorial of {number} is:", factorial(number))
	print("\n")
  

\end{lstlisting}

\begin{verbatim}
	Output:

	Rishabh Rathore
	0827CI211155
	The factorial of 5 is: 120
	

\end{verbatim}


\newpage


\qs{}{Write a Python function to check whether a number falls within a given range.}
\sol 
\begin{lstlisting}[language=Python]
	print("Rishabh Rathore")
	print("0827CI211155")
	def check_range(number, start, end):
    """
    Function to check whether a number falls within a given range.

    Parameters:
    number: The number to check.
    start: The start of the range (inclusive).
    end: The end of the range (inclusive).

    Returns:
    bool: True if the number falls within the range, False otherwise.
    """
    return start <= number <= end

	# Example usage:
	num = 7
	start_range = 5
	end_range = 10

	if check_range(num, start_range, end_range):
		print(f"{num} falls within the range [{start_range}, {end_range}]")
	else:
		print(f"{num} does not fall within the range [{start_range},
		 {end_range}]")
  

\end{lstlisting}

\begin{verbatim}
	Output:

	Rishabh Rathore
	0827CI211155
	7 falls within the range [5, 10]


\end{verbatim}


\newpage


\qs{}{Write a Python function that accepts a string and counts the number of upper and lower case letters.}
\sol 
\begin{lstlisting}[language=Python]
	print("Rishabh Rathore")
	print("0827CI211155")
	def count_upper_lower(string):
    """
    Function to count the number of upper and lower case letters in a string.

    Parameters:
    string (str): The input string.

    Returns:
    tuple: A tuple containing the count of upper 
	case letters and lower case letters, respectively.
    """
    upper_count = 0
    lower_count = 0
    for char in string:
        if char.isupper():
            upper_count += 1
        elif char.islower():
            lower_count += 1
    return upper_count, lower_count

	# Example usage:
	input_string = "Hello World"
	upper, lower = count_upper_lower(input_string)
	print("Number of upper case letters:", upper)
	print("Number of lower case letters:", lower)

\end{lstlisting}

\begin{verbatim}
	Output:

	Rishabh Rathore
	0827CI211155
	Number of upper case letters: 2
	Number of lower case letters: 8
	
	

\end{verbatim}


\newpage


\qs{}{Write a Python function that takes a list and returns a new list with distinct elements from the first list.
sample list:[1,2,3,3,3,3,4,5],Unique list:[1,2,3,4,5]}
\sol 
\begin{lstlisting}[language=Python]
	print("Rishabh Rathore")
	print("0827CI211155")
	def distinct_elements(input_list):
    """
    Function to return a new list with distinct elements from the input list.

    Parameters:
    input_list (list): The input list.

    Returns:
    list: A new list with distinct elements.
    """
    return list(set(input_list))

	# Example usage:
	original_list = [1, 2, 2, 3, 3, 4, 5, 5]
	distinct_list = distinct_elements(original_list)
	print("Original List:", original_list)
	print("Distinct List:", distinct_list)
  

\end{lstlisting}

\begin{verbatim}
	Output:

	Rishabh Rathore
	0827CI211155
	Original List: [1, 2, 2, 3, 3, 4, 5, 5]
	Distinct List: [1, 2, 3, 4, 5]


\end{verbatim}


\newpage


\qs{}{Write a Python function that takes a number as a parameter and checks whether the numberis prime or not}
\sol 
\begin{lstlisting}[language=Python]
	print("Rishabh Rathore")
	print("0827CI211155")
	def is_prime(number):
    """
    Function to check whether a number is prime or not.

    Parameters:
    number (int): The number to check.

    Returns:
    bool: True if the number is prime, False otherwise.
    """
    if number <= 1:
        return False
    elif number <= 3:
        return True
    elif number % 2 == 0 or number % 3 == 0:
        return False
    else:
        i = 5
        while i * i <= number:
            if number % i == 0 or number % (i + 2) == 0:
                return False
            i += 6
        return True

	# Example usage:
	num = 19
	if is_prime(num):
		print(f"{num} is a prime number.")
	else:
		print(f"{num} is not a prime number.")
	print("\n")
\end{lstlisting}

\begin{verbatim}
	Output:

	Rishabh Rathore
	0827CI211155
	19 is a prime number.
	

\end{verbatim}


\newpage


\qs{}{Write a Python program to access a function inside a function}
\sol 
\begin{lstlisting}[language=Python]
	print("Rishabh Rathore")
	print("0827CI211155")
	def outer_function():
    """
    Outer function.
    """
		print("This is the outer function.")

		def inner_function():
			"""
			Inner function.
			"""
			print("This is the inner function.")

    # Call the inner function
    inner_function()

	# Call the outer function
	outer_function()
  

\end{lstlisting}

\begin{verbatim}
	Output:

	Rishabh Rathore
	0827CI211155
	This is the outer function.
	This is the inner function.
	
	

\end{verbatim}


\newpage


\qs{}{Write a Python function that checks whether a passed string is a palindrome or not.}
\sol 
\begin{lstlisting}[language=Python]
	print("Rishabh Rathore")
	print("0827CI211155")
	def is_palindrome(string):
    """
    Function to check whether a passed string is a palindrome or not.

    Parameters:
    string (str): The string to check.

    Returns:
    bool: True if the string is a palindrome, False otherwise.
    """
    # Convert the string to lowercase and remove spaces
    clean_string = string.lower().replace(" ", "")

    # Compare the string with its reverse
    return clean_string == clean_string[::-1]

	# Example usage:
	input_string = "A man a plan a canal Panama"
	if is_palindrome(input_string):
		print(f"'{input_string}' is a palindrome.")
	else:
		print(f"'{input_string}' is not a palindrome.")
  

\end{lstlisting}

\begin{verbatim}
	Output:

	Rishabh Rathore
	0827CI211155
	'A man a plan a canal Panama' is a palindrome.
	
\end{verbatim}


\newpage


\qs{}{Write a python function to create and print a list where the values are the squares of numbers between 1 and 30}
\sol 
\begin{lstlisting}[language=Python]
	print("Rishabh Rathore")
	print("0827CI211155")
	def squares_list():
    """
    Function to create and print a list where the values 
	are the squares of numbers between 1 and 30.

    Returns:
    list: A list containing the squares of numbers between 1 and 30.
    """
    squares = [x ** 2 for x in range(1, 31)]
    return squares

	# Example usage:
	squares = squares_list()
	print("List of squares of numbers between 1 and 30:", squares)
  

\end{lstlisting}

\begin{verbatim}
	Output:

	Rishabh Rathore
	0827CI211155
	List of squares of numbers between 1 and 30: [1, 4, 9, 16, 25, 36, 49, 64, 81,
	100, 121, 144, 169, 196, 225, 256, 289, 324, 361, 400, 441,
	484, 529, 576, 625, 676, 729, 784, 841, 900]
	

\end{verbatim}


\newpage


\qs{}{Write a Python program to print the even numbers from a given list}
\sol 
\begin{lstlisting}[language=Python]
	print("Rishabh Rathore")
	print("0827CI211155")
	def print_even_numbers(input_list):
    """
    Function to print the even numbers from a given list.

    Parameters:
    input_list (list): The input list.

    Returns:
    None
    """
    even_numbers = [num for num in input_list if num % 2 == 0]
    print("Even numbers from the given list:", even_numbers)

	# Example usage:
	numbers = [1, 2, 3, 4, 5, 6, 7, 8, 9, 10]
	print_even_numbers(numbers)
  

\end{lstlisting}

\begin{verbatim}
	Output:

	Rishabh Rathore
	0827CI211155
	Even numbers from the given list: [2, 4, 6, 8, 10]
	

\end{verbatim}


\newpage


\qs{}{Write a program to create a class student with data member name. roll no. Semester. and display the data using object}
\sol 
\begin{lstlisting}[language=Python]
	print("Rishabh Rathore")
	print("0827CI211155")
	class Student:
    """
    Class representing a student.
    """
    def __init__(self, name, roll_no, semester):
        """
        Constructor to initialize the data members of the Student class.

        Parameters:
        name (str): The name of the student.
        roll_no (str): The roll number of the student.
        semester (str): The semester of the student.
        """
        self.name = name
        self.roll_no = roll_no
        self.semester = semester

    def display_data(self):
        """
        Method to display the data of the student.
        """
        print("Name:", self.name)
        print("Roll No:", self.roll_no)
        print("Semester:", self.semester)

	# Create an object of the Student class
	student1 = Student("Ramesh Kumar", "12345", "Spring 2024")

	# Display the data using the object
	print("Student Data:")
	student1.display_data()

\end{lstlisting}

\begin{verbatim}
	Output:

	Rishabh Rathore
	0827CI211155
	Student Data:
	Name: Ramesh Kumar
	Roll No: 12345
	Semester: Spring 2024
	

\end{verbatim}



\newpage
% TODO solve the red line problem %

\qs{Write a program to demonstrate the use of init }
\sol
\begin{lstlisting}[language=Python]

	print("Rishabh Rathore")
	print("0827CI211155")
	class Car:
    """
    Class representing a car.
    """
    def __init__(self, make, model, year):
        """
        Constructor to initialize the data members of the Car class.

        Parameters:
        make (str): The make of the car.
        model (str): The model of the car.
        year (int): The manufacturing year of the car.
        """
        self.make = make
        self.model = model
        self.year = year
        self.odometer_reading = 0  # Additional attribute

    def get_car_info(self):
        """
        Method to display information about the car.
        """
        car_info = f"{self.year} {self.make} {self.model}"
        return car_info

    def read_odometer(self):
        """
        Method to read the odometer reading of the car.
        """
        print(f"This car has {self.odometer_reading} miles on it.")

	# Create an object of the Car class
	my_car = Car("Toyota", "Corolla", 2022)

	# Display information about the car
	print("Car Information:", my_car.get_car_info())

	# Read the odometer reading of the car
	my_car.read_odometer()
  

\end{lstlisting}

\begin{verbatim}
	Output:

	Rishabh Rathore
	0827CI211155
	Car Information: 2022 Toyota Corolla
	This car has 0 miles on it.
	

\end{verbatim}


\newpage


\qs{}{Write a program for Inheritence.a) single level b) Multiple c) Multilevel d) Hybrid e) Hierarchical}
\sol 
\begin{lstlisting}[language=Python]
	print("Rishabh Rathore")
	print("0827CI211155")

	# Single level inheritance
	class Parent:
		def parent_method(self):
			print("This is the parent method.")
	
	class Child(Parent):
		def child_method(self):
			print("This is the child method.")
	
	# Single level inheritance
	child_obj = Child()
	child_obj.parent_method()
	child_obj.child_method()
	
	# Multiple inheritance
	class Parent1:
		def method1(self):
			print("This is method 1 of Parent1.")
	
	class Parent2:
		def method2(self):
			print("This is method 2 of Parent2.")
	
	class Child1(Parent1, Parent2):
		def child_method1(self):
			print("This is the child1 method.")
	
	# Multiple inheritance
	child_obj1 = Child1()
	child_obj1.method1()
	child_obj1.method2()
	child_obj1.child_method1()
	
	# Multilevel inheritance
	class Grandparent:
		def grandparent_method(self):
			print("This is the grandparent method.")
	
	class Parent3(Grandparent):
		def parent_method2(self):
			print("This is the parent method.")
	
	class Child2(Parent3):
		def child_method2(self):
			print("This is the child method.")
	
	# Multilevel inheritance
	child_obj2 = Child2()
	child_obj2.grandparent_method()
	child_obj2.parent_method2()
	child_obj2.child_method2()
	
	# Hybrid inheritance
	class Parent4:
		def method1(self):
			print("This is method 1 of Parent4.")
	
	class Parent5:
		def method2(self):
			print("This is method 2 of Parent5.")
	
	class Parent6:
		def method3(self):
			print("This is method 3 of Parent6.")
	
	class Child3(Parent4, Parent5, Parent6):
		def child_method3(self):
			print("This is the child method.")
	
	# Hybrid inheritance
	child_obj3 = Child3()
	child_obj3.method1()
	child_obj3.method2()
	child_obj3.method3()
	child_obj3.child_method3()
	
	# Hierarchical inheritance
	class Parent7:
		def parent_method4(self):
			print("This is the parent method.")
	
	class Child4(Parent7):
		def child1_method4(self):
			print("This is the child1 method.")
	
	class Child5(Parent7):
		def child2_method4(self):
			print("This is the child2 method.")
	
	# Hierarchical inheritance
	child1_obj = Child4()
	child1_obj.parent_method4()
	child1_obj.child1_method4()
	
	child2_obj = Child5()
	child2_obj.parent_method4()
	child2_obj.child2_method4()
\end{lstlisting}

\begin{verbatim}
	Output:

	Rishabh Rathore
	0827CI211155
	This is the parent method.
	This is the child method.
	This is method 1 of Parent1.
	This is method 2 of Parent2.
	This is the child1 method.
	This is the grandparent method.
	This is the parent method.
	This is the child method.
	This is method 1 of Parent4.
	This is method 2 of Parent5.
	This is method 3 of Parent6.
	This is the child method.
	This is the parent method.
	This is the child1 method.
	This is the parent method.
	This is the child2 method.

	

\end{verbatim}


\newpage



\qs{}{ Write a program to demonstrate overriding }
\sol 
\begin{lstlisting}[language=Python]
	print("Rishabh Rathore")
	print("0827CI211155")
	class Parent:
    def show_message(self):
        print("This is the parent message.")

	class Child(Parent):
		def show_message(self):
			print("This is the overridden message.")

	# Create objects of both Parent and Child classes
	parent_obj = Parent()
	child_obj = Child()

	# Call the method on each object
	print("Calling show_message() method of Parent class:")
	parent_obj.show_message()

	print("\nCalling show_message() method of Child class:")
	child_obj.show_message()

\end{lstlisting}

\begin{verbatim}
	Output:

	Rishabh Rathore
	0827CI211155
	Calling show_message() method of Parent class:
	This is the parent message.
	
	Calling show_message() method of Child class:
	This is the overridden message.

\end{verbatim}


\newpage



\end{document}
